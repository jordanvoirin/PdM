%%%%%%%%%%%%%%%%%%%%%%%%%%%%%%%%%%%%%%%%%
% Masters/Doctoral Thesis 
% LaTeX Template.
% Version 2.5 (27/8/17)
%
% This template was downloaded from:
% http://www.LaTeXTemplates.com
%
% Version 2.x major modifications by:
% Vel (vel@latextemplates.com)
%
% This template is based on a template by:
% Steve Gunn (http://users.ecs.soton.ac.uk/srg/softwaretools/document/templates/)
% Sunil Patel (http://www.sunilpatel.co.uk/thesis-template/)
%
% Template license:
% CC BY-NC-SA 3.0 (http://creativecommons.org/licenses/by-nc-sa/3.0/)
%
%%%%%%%%%%%%%%%%%%%%%%%%%%%%%%%%%%%%%%%%%

%----------------------------------------------------------------------------------------
%	PACKAGES AND OTHER DOCUMENT CONFIGURATIONS
%----------------------------------------------------------------------------------------

\documentclass[11pt,english,singlespacing,headsepline]{MastersDoctoralThesis} % The class file specifying the document structure

\usepackage[utf8]{inputenc} % Required for inputting international characters
\usepackage[T1]{fontenc} % Output font encoding for international characters

\usepackage{mathpazo} % Use the Palatino font by default

\usepackage[backend=bibtex,style=authoryear,natbib=true]{biblatex} % Use the bibtex backend with the authoryear citation style (which resembles APA)

\addbibresource{biblio.bib} % The filename of the bibliography

\usepackage[autostyle=true]{csquotes} % Required to generate language-dependent quotes in the bibliography

%----------------------------------------------------------------------------------------
%	MARGIN SETTINGS
%----------------------------------------------------------------------------------------

\geometry{
	paper=a4paper, % Change to letterpaper for US letter
	inner=2.5cm, % Inner margin
	outer=3.8cm, % Outer margin
	bindingoffset=.5cm, % Binding offset
	top=1.5cm, % Top margin
	bottom=1.5cm, % Bottom margin
	%showframe, % Uncomment to show how the type block is set on the page
}

%----------------------------------------------------------------------------------------
%	THESIS INFORMATION
%----------------------------------------------------------------------------------------

\thesistitle{Evaluation of Optical Aberrations using Phase Diversity} % Your thesis title, this is used in the title and abstract, print it elsewhere with \ttitle
\supervisor{Dr. Laurent \textsc{Jolissaint}} % Your supervisor's name, this is used in the title page, print it elsewhere with \supname
\cosupervisor{Dr. Jean-Paul \textsc{Kneib}}
\examiner{} % Your examiner's name, this is not currently used anywhere in the template, print it elsewhere with \examname
\degree{Master in Applied Physics} % Your degree name, this is used in the title page and abstract, print it elsewhere with \degreename
\author{Jordan \textsc{Voirin}} % Your name, this is used in the title page and abstract, print it elsewhere with \authorname
\addresses{} % Your address, this is not currently used anywhere in the template, print it elsewhere with \addressname

\subject{Optical Physics} % Your subject area, this is not currently used anywhere in the template, print it elsewhere with \subjectname
\keywords{} % Keywords for your thesis, this is not currently used anywhere in the template, print it elsewhere with \keywordnames
\university{\href{http://www.epfl.ch}{Ecole Polytechnique Fédérale de Lausanne}} % Your university's name and URL, this is used in the title page and abstract, print it elsewhere with \univname
\department{\href{http://sb.epfl.ch}{Basic Sciences}} % Your department's name and URL, this is used in the title page and abstract, print it elsewhere with \deptname
\group{\href{http://lastro.epfl.ch/}{Astrophysics laboratory}} % Your research group's name and URL, this is used in the title page, print it elsewhere with \groupname
\faculty{\href{http://sb.epfl.ch/physics}{Physics}} % Your faculty's name and URL, this is used in the title page and abstract, print it elsewhere with \facname

\AtBeginDocument{
\hypersetup{pdftitle=\ttitle} % Set the PDF's title to your title
\hypersetup{pdfauthor=\authorname} % Set the PDF's author to your name
\hypersetup{pdfkeywords=\keywordnames} % Set the PDF's keywords to your keywords
}

%parameter to type code

\definecolor{forestgreen}{rgb}{0.0, 0.27, 0.13}
\lstset{frame=tb,
  language=Python,
  aboveskip=3mm,
  belowskip=3mm,
  showstringspaces=false,
  columns=flexible,
  basicstyle={\small\ttfamily},
  numbers=left, numbersep=5pt,
  numberstyle=\tiny\color{gray},
  keywordstyle=\color{blue},
  commentstyle=\color{forestgreen},
  stringstyle=\color{magenta},
  breaklines=true,
  breakatwhitespace=true,
  tabsize=3
}

%for description list to be aligned with the text
\setlist[description]{leftmargin=0cm}

\begin{document}

\frontmatter % Use roman page numbering style (i, ii, iii, iv...) for the pre-content pages

\pagestyle{plain} % Default to the plain heading style until the thesis style is called for the body content

%----------------------------------------------------------------------------------------
%	TITLE PAGE
%----------------------------------------------------------------------------------------

\begin{titlepage}
\begin{center}

%\vspace*{.05\textheight}
\includegraphics[scale=.25]{Logo/Logo_EPFL.png} \hfill
\includegraphics[scale=.625]{Logo/HEIG-VD_Logo.png} \\[1cm]

{\scshape\LARGE \univname\par}\vspace{1.2cm} % University name
\textsc{\Large Master Thesis}\\[0.5cm] % Thesis type

\HRule \\[0.4cm] % Horizontal line
{\huge \bfseries \ttitle\par}\vspace{0.4cm} % Thesis title
\HRule \\[1.5cm] % Horizontal line
 
\begin{minipage}[t]{0.4\textwidth}
\begin{flushleft} \large
\emph{Author:}\\
{\authorname} % Author name - remove the \href bracket to remove the link
\end{flushleft}
\end{minipage}
\begin{minipage}[t]{0.4\textwidth}
\begin{flushright} \large
\emph{Supervisors:} \\
\href{https://www.researchgate.net/profile/Laurent_Jolissaint/publications}{\supname}\\ % Supervisor name - remove the \href bracket to remove the link  
\href{https://people.epfl.ch/cgi-bin/people?id=222189&op=bio}{\cosupname}
\end{flushright}
\end{minipage}\\[3cm]
 
\vfill

\large \textit{A thesis submitted in fulfillment of the requirements\\ for the degree of \degreename}\\[0.3cm] % University requirement text
\textit{in the}\\[0.4cm]
\groupname\\\deptname\\[2cm] % Research group name and department name
 
\vfill

{\large \today}\\[3cm] % Date
%\includegraphics{Logo} % University/department logo - uncomment to place it
 
\vfill
\end{center}
\end{titlepage}

%----------------------------------------------------------------------------------------
%	DECLARATION PAGE
%----------------------------------------------------------------------------------------

\begin{declaration}
\addchaptertocentry{\authorshipname} % Add the declaration to the table of contents
\noindent I, \authorname, declare that this thesis titled, \enquote{\ttitle} and the work presented in it are my own. I confirm that:

\begin{itemize} 
\item This work was done wholly or mainly while in candidature for a research degree at this University.
\item Where any part of this thesis has previously been submitted for a degree or any other qualification at this University or any other institution, this has been clearly stated.
\item Where I have consulted the published work of others, this is always clearly attributed.
\item Where I have quoted from the work of others, the source is always given. With the exception of such quotations, this thesis is entirely my own work.
\item I have acknowledged all main sources of help.
\item Where the thesis is based on work done by myself jointly with others, I have made clear exactly what was done by others and what I have contributed myself.\\
\end{itemize}
 
\noindent Signed:\\
\rule[0.5em]{25em}{0.5pt} % This prints a line for the signature
 
\noindent Date:\\
\rule[0.5em]{25em}{0.5pt} % This prints a line to write the date
\end{declaration}

\cleardoublepage

%----------------------------------------------------------------------------------------
%	QUOTATION PAGE
%----------------------------------------------------------------------------------------

\vspace*{0.2\textheight}

\noindent\enquote{\itshape For the sake of persons of ... different types, scientific truth should be presented in different forms, and should be regarded as equally scientific, whether it appears in the robust form and the vivid coloring of a physical illustration, or in the tenuity and paleness of a symbolic expression. }\bigbreak

\hfill James Clerk Maxwell

%----------------------------------------------------------------------------------------
%	ABSTRACT PAGE
%----------------------------------------------------------------------------------------

\begin{abstract}
\addchaptertocentry{\abstractname} % Add the abstract to the table of contents
This thesis deals with the aberrations present in an optical system and their corrections. Indeed, nowadays the telescopes are so big that they are limited by the Earth's atmosphere turbulence and their own defects.  Sophisticated method are developed to take into account the aberrations and correct them. We study a retrieval method called phase diversity that uses the image given by the system to correct the aberrations present on the optical path. 

The first part of our work is the test and use of a phase diversity algorithm developed by \citet{mugnier_2006} at ONERA. The algorithm regarding the noise level in the images and the different parameters that could influence, behaves adequately. However, the usage of the phase diversity in comparison with a simulated calibrated aberration and another wavefront sensor highlighted problems with our optical bench and its simulation.

The second part presents the development and test of our phase diversity algorithm. The algorithm is based on a analytical approach unlike the ONERA phase diversity, which is based on a statistical approach. Our algorithm gives good results, however it shows a limited validity domain. A recursive approach is then proposed and tested to increase the validity domain with success up to an RMS wavefront error of $\sim\frac{\lambda}{4}$.

In conclusion, we tested the ONERA phase diversity algorithm successful. But its use highlighted technical problems of our setup. We then developed our own phase diversity algorithm in order to have a method that we fully understand and control. Using a recursive approach we are able to increase by a factor three the validity domain.
\end{abstract}

%----------------------------------------------------------------------------------------
%	ACKNOWLEDGEMENTS
%----------------------------------------------------------------------------------------

\begin{acknowledgements}
\addchaptertocentry{\acknowledgementname} % Add the acknowledgements to the table of contents
I would first like to thank my thesis supervisor Dr. Laurent Jolissaint of the industrial automation institute at the School of Management and Engineering (HEIG-VD) of Yverdon-Les-Bains. 

I would also like to thank the expert : Dr. Jean-Paul Kneib, Director of the astrophysics laboratory of the Ecole Polytechnique Federale de Lausanne (EPFL). Without his passionate participation and input, the validation survey could not have been successfully conducted.

I would also like to thank my colleagues Audrey Bouxin, Scientific collaborator at HEIG-VD and Jeremie Baudet, Scientific collaborator at HEIG-VD for their precious help and inputs during this project.

Finally, I must express my very profound gratitude to my family and to my girlfriend for providing me with unfailing support and continuous encouragement throughout my years of study and through the process of researching and writing this thesis. This accomplishment would not have been possible without them. Thank you.
\end{acknowledgements}

%----------------------------------------------------------------------------------------
%	LIST OF CONTENTS/FIGURES/TABLES PAGES
%----------------------------------------------------------------------------------------

\tableofcontents % Prints the main table of contents

\listoffigures % Prints the list of figures

%\listoftables % Prints the list of tables

%----------------------------------------------------------------------------------------
%	ABBREVIATIONS
%----------------------------------------------------------------------------------------

\begin{abbreviations}{ll} % Include a list of abbreviations (a table of two columns)
\textbf{CCD} & \textbf{Charge}-\textbf{Coupled} \textbf{D}evice \\
\textbf{CMOS} & \textbf{C}omplementary \textbf{M}etal–\textbf{O}xide–\textbf{S}emiconductor \\
\textbf{DAG} & \textbf{D}ogu \textbf{A}nadolu \textbf{G}ozlemevi, Eastern Anatolia Observatory \\
\textbf{ELT} & \textbf{E}xtremly \textbf{L}arge \textbf{T}elescope \\
\textbf{FWHM} & \textbf{F}ull \textbf{W}idth \textbf{H}alf \textbf{M}aximum\\
\textbf{GUI} & \textbf{G}raphic \textbf{U}ser \textbf{I}nterface \\
\textbf{HEIG-VD} & \textbf{H}aute \textbf{E}cole d'\textbf{I}ngéniérie et de \textbf{G}estion du canton de \textbf{V}au\textbf{D}\\
 & School of Management and Engineering \\
\textbf{IR} & \textbf{I}mpulse \textbf{R}esponse \\
\textbf{JMAP} & \textbf{J}oint \textbf{M}aximum \textbf{\textit{A}} \textbf{\textit{P}}\textit{osteriori} \\
\textbf{NCPA} & \textbf{N}on \textbf{C}ommon \textbf{P}ath \textbf{A}berration \\
\textbf{ONERA} & \textbf{O}ffice \textbf{Nationale} d'\textbf{E}tude et de \textbf{R}echerce \textbf{A}erospatiale \\
 & French National Aerospace Research Center \\
\textbf{OTF} & \textbf{O}ptical \textbf{T}ransfer \textbf{F}unction \\
\textbf{P2V} & \textbf{P}eak \textbf{T}o \textbf{V}alley  \\
\textbf{PSF} & \textbf{P}oint \textbf{S}pread \textbf{F}unction \\
\textbf{RMS} & \textbf{R}oot \textbf{M}ean \textbf{S}quare\\
\textbf{RMSE}& \textbf{R}oot \textbf{M}ean \textbf{S}quare \textbf{Error} \\
\textbf{VLT} & \textbf{V}ery \textbf{L}arge \textbf{T}elescope \\
\textbf{WFS} & \textbf{W}ave\textbf{F}ront \textbf{S}ensor \\

\end{abbreviations}

%----------------------------------------------------------------------------------------
%	PHYSICAL CONSTANTS/OTHER DEFINITIONS
%----------------------------------------------------------------------------------------

%\begin{constants}{lr@{${}={}$}l} % The list of physical constants is a three column table
%
%% The \SI{}{} command is provided by the siunitx package, see its documentation for instructions on how to use it
%
%Speed of Light & $c_{0}$ & \SI{2.99792458e8}{\meter\per\second} (exact)\\
%%Constant Name & $Symbol$ & $Constant Value$ with units\\
%
%\end{constants}

%----------------------------------------------------------------------------------------
%	SYMBOLS
%----------------------------------------------------------------------------------------

%\begin{symbols}{lll} % Include a list of Symbols (a three column table)
%
%$a$ & distance & \si{\meter} \\
%$P$ & power & \si{\watt} (\si{\joule\per\second}) \\
%%Symbol & Name & Unit \\
%
%\addlinespace % Gap to separate the Roman symbols from the Greek
%
%$\omega$ & angular frequency & \si{\radian} \\
%
%\end{symbols}

%----------------------------------------------------------------------------------------
%	DEDICATION
%----------------------------------------------------------------------------------------

%\dedicatory{For/Dedicated to/To my\ldots} 

%----------------------------------------------------------------------------------------
%	THESIS CONTENT - CHAPTERS
%----------------------------------------------------------------------------------------

\mainmatter % Begin numeric (1,2,3...) page numbering

\pagestyle{thesis}

% Chapter 1

\chapter*{Introduction} % Main chapter title
\addcontentsline{toc}{chapter}{Introduction}
\label{Introduction} % For referencing the chapter elsewhere, use \ref{Chapter1} 

%----------------------------------------------------------------------------------------

% Define some commands to keep the formatting separated from the content 
\newcommand{\keyword}[1]{\textbf{#1}}
\newcommand{\tabhead}[1]{\textbf{#1}}
\newcommand{\code}[1]{\texttt{#1}}
\newcommand{\file}[1]{\texttt{\bfseries#1}}
\newcommand{\option}[1]{\texttt{\itshape#1}}

%----------------------------------------------------------------------------------------

This master thesis treats of one of the major challenge in astronomy and in optics in general, the aberrations and their corrections. We will study and develop a method to identify the static aberrations present in an  optical system using the data produced by it.

\vspace{1cm}


Ground based observation is as old as the first men. But since the Galilei Galileo telescope in 1609, the resolution power of the telescopes is far greater. Many inventions lead to the construction of the largest telescope ever built, the Very Large Telescope (VLT). Now that we are able to build enormous telescopes with mirror diameter up to 8m, soon to be bitten by the Extremely Large Telescope (ELT) with its 18m, the limitation of the resolution is not the diameter anymore, but the atmosphere. Indeed, the turbulence present in the atmosphere creates a turbulent distribution of refractive index due to the distribution of temperature principally. This alters the optical path of the light through the Earth's atmosphere which finally deteriorates the images given by the telescope.

\begin{figure}
\begin{center}
\includegraphics[width=0.6\textwidth,angle=0]{Figures/ao_scheme.jpg}
\caption{Adaptive optics schema}
source : \url{http://www.bo.astro.it/ter5/ter5/ao.html}.
\label{fig:ao_scheme}
\end{center}
\end{figure}

The first system to correct aberrations in telescopes is the active optics system. It consisted of actuators placed under the telescope mirrors capable of correcting the telescope deformation under the gravity effect, vibration, etc... Those systems are not fast enough to correct for the turbulence but they already improve the images. The state-of-the-art technique is the adaptive optics, developed to take into account the effects of the atmosphere. This technique characterizes the aberrations present in the wavefront and correct them in real time. The schema of principle is shown in Figure \ref{fig:ao_scheme}. It uses a wavefront sensor to characterize its deformation and then passes the information to a deformable mirror to correct it, it is called an adaptive optic loop. It works well, but the drawback of this method is that it does not correct the aberrations up to the scientific detector. Those uncorrected aberrations are called Non-Common Path Aberration (NCPA), they are generally static or slowly evolving over time. The adaptive optic also requires a complex optical system before the scientific detector. The beam is split in two. One arm goes to the scientific detector and the other is used in the adaptive optic loop.

We present a method called phase diversity that uses images of a scientific detector to deduct the aberration present in the wavefront. In our case, we work on the static aberrations that deform the wavefront. The phase diversity was first introduced by \citet{Gonsalves_1982}. The idea is to use one focused and one defocused image to retrieve the aberrations. Two images at least are needed due to the property of light acquisition. There is an indetermination, because the detector is only sensitive to the intensity of light and not the light wave itself. It is raised adding a phase diversity. Unlike adaptive optic system, the phase diversity requires nearly no additional optical component depending on how it is implemented. The final aim of this project would be to integrate the developed phase diversity into the design of the DAG telescope in Turkey, see Figure \ref{fig:DAGtelescope}.

\begin{figure}
\begin{center}
\includegraphics[width=0.5\textwidth,angle=0]{Figures/DAGtelescope.jpg} \\
\caption{DAG 4m telescope} 
source : \url{http://www.eie.it/en/progetti/dag-dogu-anadolu-gozlemevi}.
\label{fig:DAGtelescope}
\end{center}
\end{figure}

\vspace{1cm}

This report is organized as following. First, a review of the necessary theoretical background is reminded, going from the scalar theory of diffraction to the description of phase retrievals methods. Then we present an experiment performed at the optic laboratory at HEIG-VD. The ONERA (Office National d'Etudes et de Recherches Aerospatiales) algorithm of phase diversity is tested and used in comparison with a Shack-Hartmann wavefront sensor. And finally, the development of an analytical phase diversity algorithm is presented and tested with simulated Point Spread Functions (PSFs).

\chapter{Theoretical background} 
\label{ch:THback}

In this chapter, we will present the theory upon which this work is based. First, the light propagation formalism will be reminded through the scalar diffraction theory based on the \citet{goodman_1968}. Then we will describe in general an imaging system and its properties. And finally we will discuss the wavefront aberration theory.

\section{Scalar Diffraction Theory}
\label{sec:ScaDifTh}

\subsection{Scalar Field and Helmholtz equation}
\label{subsec:ScalF_HelmEqt}

A monochromatic wave, at position $P$ and time $t$, can be represented by a scalar field $u(P,t)$ written as :

\begin{equation}
u(P,t) =  A(P) exp\left[-j2\pi\nu t + j\phi(P)\right],
\label{eqt:WvscalarField}
\end{equation}

where $A(P)$ and $\phi(P)$ are the amplitude and phase, respectively, of the wave at position P and $\nu$ is the wave frequency.

The spatial part of eqt. \eqref{eqt:WvscalarField}, also called  phasor  in the literature, 

\begin{equation}
U(P) = A(P)e^{j\phi(P)},
\label{eqt:phasor}
\end{equation}

must verify the Helmotz equation : 

\begin{equation}
(\nabla^2 + k^2)U = 0,
\label{eqt:HelmholtzEqt}
\end{equation}

where $k$ is the wave number given by

\begin{equation}
k = 2\pi n \frac{\nu}{c} = \frac{2\pi}{\lambda},
\label{eqt:wavenumber}
\end{equation}

and $\lambda$ is the wavelength in the dielectric medium.


\subsection{Rayleigh-Sommerfeld integral}
\label{subsec:Ray_Som_int}

\begin{figure}
\centering
    \begin{subfigure}{0.4\textwidth}
        \includegraphics[width=\textwidth]{Figures/Ray_Som_Diff}
        \caption{Rayleigh-Sommerfeld formulation of diffraction by a plane screen, \citep[Chapter 3.5]{goodman_1968}.}
        \label{subfig:Ray_Som_Diff}
    \end{subfigure}
    \quad
    \begin{subfigure}{0.5\textwidth}
        \includegraphics[width=\textwidth]{Figures/Diff_Geom}
        \caption{Diffraction geometry, \citep[Chapter 4.1]{goodman_1968}.}
        \label{subfig:Diff_Geom}
    \end{subfigure}
    \decoRule
    \caption{Diffraction Schemas}
    \label{fig:Diff_Schemas}
\end{figure}

Rayleigh and Sommerfeld developed a formalism using the Helmholtz equation and Green's Theorem to compute the induced diffraction by a plane screen. Let's suppose that we have a monochromatic source at $\widetilde{P_0}$ on the left of a plane screen with aperture $\Sigma$, the Rayleigh-Sommerfeld formula allows to compute the complex amplitude at $P_0$ on the right of the plane screen (see Figure \ref{subfig:Ray_Som_Diff}).

\begin{equation}
U(P_0) = \frac{1}{j\lambda} \iint\limits_{\Sigma} U'(P_1)\frac{exp(jkr_{01})}{r_{01}}cos(\mathbf{n},\mathbf{r_{01}})\mathbf{ds}
\label{eqt:Ray_Som_Formula}
\end{equation}

$U'(P_1)$ is the complex amplitude on the screen, $cos(\mathbf{n},\mathbf{r_{01}})$ is the cosine of the angle between the aperture plane normal toward the source and the vector $\mathbf{r_{01}} = \mathbf{P_0P_1}$ given by 

\begin{equation}
r_{01} = \sqrt{z^2 + (x-\xi)^2 + (y-\eta)^2}.
\label{eqt:r_01}
\end{equation}

We can rewrite eqt. \eqref{eqt:Ray_Som_Formula} using $cos(\mathbf{n},\mathbf{r_{01}}) = cos(\theta) = \frac{z}{r_{01}}$ and the coordinate systems ($\xi,\eta$) and ($x,y$), see Figure \ref{subfig:Diff_Geom},

\begin{equation}
U(x,y) = \frac{z}{j\lambda} \iint\limits_{-\infty}^{\infty} U(\xi,\eta)\frac{exp(jkr_{01})}{r_{01}^2} d\xi d\eta.
\label{eqt:Ray_Som_formula_xy_en}
\end{equation}

We can integrate from $-\infty$ to $\infty$, using $U(\xi,\eta) = P(\xi,\eta)U'(\xi,\eta)$ where $P(\xi,\eta)$ is the pupil function. The latter equals to one in the pupil and zero outside. 

\subsection{Fresnel approximation}
\label{subsec:FresnelApprox}

To reduce eqt. \eqref{eqt:Ray_Som_formula_xy_en}, also known as the Huygens-Fresnel principle, one can approximate the distance $r_{01}$ using the taylor expansion of the square root :

\begin{equation}
r_{01} = z \sqrt{1 + \frac{x-\xi}{z} + \frac{y-\eta}{z}} \approx z \left[1+\frac{1}{2}\left(\frac{x-\xi}{z}\right)^2+\frac{1}{2}\left(\frac{y-\eta}{z}\right)^2\right]
\label{eqt:approx_r01}
\end{equation}

To obtain the Fresnel approximation, one has to replace $r_{01}$ by eqt. \eqref{eqt:approx_r01} in eqt. \eqref{eqt:Ray_Som_formula_xy_en}. At the denominator, only the first term $z$ is kept, since the introduced error is small, but in the exponential everything is kept. Then the final expression is given by,

\begin{equation}
U(x,y) = \frac{e^{jkz}}{j\lambda z} \iint\limits_{-\infty}^{\infty} U(\xi,\eta)exp \left\lbrace j \frac{k}{2z}\left[(x-\xi)^2+(y-\eta)^2\right]\right\rbrace d\xi d\eta.
\label{eqt:fresnel_Approx_conv}
\end{equation}

In this form, the Fresnel approximation can be seen as a convolution between $U(\xi,\eta)$ and $h(x,y) = \frac{e^{jkz}}{j\lambda z}exp\left[\frac{jk}{2z}\left(x^2+y^2\right)\right]$.

Another form is found by developing $\left[(x-\xi)^2+(y-\eta)^2\right]$,

\begin{equation}
U(x,y) = \frac{e^{jkz}}{j\lambda z} e^{j\frac{k}{2z}(x^2+y^2)} \iint\limits_{-\infty}^{\infty} \left\lbrace U(\xi,\eta) e^{j\frac{k}{2z}(\xi^2+\eta^2)}\right\rbrace e^{-j\frac{2\pi}{\lambda z}(x\xi+y\eta)} d\xi d\eta,
\label{eqt:fresnel_Approx_FT}
\end{equation}

it is the Fourier transform of the complex field in the pupil multiplied by a quadratic phase exponential.

\subsection{Fraunhofer approximation}
\label{subsec:FraunhoferApprox}

In addition to the Fresnel approximation, we can introduce another approximation using the condition,

\begin{equation}
z >> \frac{k(\xi^2+\eta^2)_{max}}{2}.
\label{eqt:Fraun_Cond}
\end{equation}

If eqt. \eqref{eqt:Fraun_Cond} is satisfied the Fresnel approximation simplifies, since the quadratic phase factor in $(\xi,\eta)$ is approximately one on the entire pupil, as

\begin{equation}
U(x,y) = \frac{e^{jkz}}{j\lambda z} e^{j\frac{k}{2z}(x^2+y^2)} \iint\limits_{-\infty}^{\infty} U(\xi,\eta) e^{-j\frac{2\pi}{\lambda z}(x\xi+y\eta)} d\xi d\eta.
\label{eqt:Fraunhofer_approx}  
\end{equation}

For instance, at a wavelength of 637.5 nm and a pupil diameter of 3.6 mm the Fraunhofer approximation constrain $z$ to be greater than 63 meters to be valid.

\subsection{Converging lens introduction}
\label{subsec:ConvLensIntro}

The Fraunhofer conditions are severe as shown above, but one can reduce the distance $z$ by observing at the focal plane of a converging lens. Indeed, using the paraxial approximation, i.e. small angles with respect with the optical axis, the lens transmission function is given by,

\begin{align}
t_l(\xi , \eta) &= exp \left[ j k n \Delta_0 \right] exp \left[ -jk \left( n-1 \right)\frac{ \xi^2 + \eta^2 }{2}\left( \frac{1}{R_1} - \frac{1}{R_2}\right) \right] \nonumber \\ 
&= exp \left[ -j \frac{k}{2f} (\xi^2+\eta^2) \right] ,
\label{eqt:lensTl}
\end{align}

where $n$ is the refractive index of the lens material, $R_1$ and $R_2$ are the radii of curvature of the front and back surface of the lens, respectively and f is the focal length of the lens defined as,

\begin{equation}
\frac{1}{f} \equiv (n-1)\left(\frac{1}{R_1}-\frac{1}{R_2}. \right)
\label{eqt:focal_length}
\end{equation}

We can define $U_l(\xi,\eta) = U(\xi,\eta)t_l(\xi,\eta)$, which represents the complex amplitude passing through a lens. Finally, replacing $U(\xi,\eta)$ by $U_l(\xi,\eta)$ in Fresnel approximation and setting the observing distance to the focal length of the converging lens, we recover the Fraunhofer approximation,

\begin{align}
U(x,y) &= \frac{e^{jkz}}{j\lambda z} e^{j\frac{k}{2z}(x^2+y^2)}  \mathcal{F}\left\lbrace U(\xi,\eta) exp\left[j\frac{k}{2}(\xi^2+\eta^2)(\frac{1}{z}-\frac{1}{f})\right]\right\rbrace \nonumber \\
&\overset{z=f}{=} \frac{e^{jkz}}{j\lambda z} e^{j\frac{k}{2z}(x^2+y^2)}  \mathcal{F}\left\lbrace U(\xi,\eta)\right\rbrace.
\end{align}

\section{Imaging system}
\label{sec:ImSystem}

\begin{figure}
\begin{center}
\includegraphics[width=0.8\textwidth,angle=0]{Figures/ImagingInstrumentGenSchema}
\decoRule
\caption{Schema of a imaging instrument, \citep[Chapter 6.1]{goodman_1968}}
\label{fig:ImagingInstrumentGenSchema}
\end{center}
\end{figure}

An imaging system, such as a telescope, is used to acquire images of an object as perfectly as possible. An optical system,  forming an instrument, is composed by lenses, mirrors, etc... and a detector (can be the human eye). A complex optical system can be reduced to a pupil, $P(\xi,\eta)$, and a focal length, $f$. The diffraction of the wave can be determined by the Fraunhofer approximation as long as the paraxial approximation is valid, see subsection \ref{subsec:ConvLensIntro}. And the observed image of an incoherent object at the focal plane of the system is proportional to the square modulus of the complex amplitude $U(x,y)$,

%\begin{equation}
%i(x,y) \alpha |\left[\mathcal{F}\left\lbrace U(\xi,\eta) \right\rbrace\right](x,y)|^2.
%\label{eqt:i_modu_FT}
%\end{equation}

\subsection{Impulse Response (IR)}
\label{subsec:IR}

\begin{figure}
\centering
    \begin{subfigure}{0.4\textwidth}
        \includegraphics[width=\textwidth]{Figures/PSF}
        \caption{PSF}
        \label{subfig:PSF}
    \end{subfigure}
    \quad
    \begin{subfigure}{0.4\textwidth}
        \includegraphics[width=\textwidth]{Figures/PSFzoom}
        \caption{PSF zoom x10}
        \label{subfig:PSFzoom}
    \end{subfigure}
    \decoRule
    \caption{PSF of a perfect imaging system composed by a 3.6 mm pupil and a focal length of 80 mm at a wavelength of 637.5 nm. The size, N, of the PSF is 400 and the pixel size is 5.3 $\mu m$.}
    \label{fig:PSF}
\end{figure} 

The impulse response or point spread function (PSF), $h(x,y;u,v)$, of an optical system is  the field amplitude induced at coordinates $(x,y)$ by a unit-amplitude point source at object coordinates $(u,v)$. Using the linearity of the wave propagation, we can write the imaged amplitude as the superposition integral,

\begin{equation}
U(x,y) = \iint_{-\infty}^{\infty} h(x,y;u,v)U(u,v)dudv
\label{eqt:superpositionIntegral}
\end{equation}

\subsection{Optical Transfer Function (OTF)}
\label{subsec:OTF}

\begin{figure}
\begin{center}
\includegraphics[width=0.5\textwidth,angle=0]{Figures/OTF}
\decoRule
\caption{OTF of a perfect imaging system composed by a 3.6 mm pupil and a focal length of 80 mm at a wavelength of 637.5 nm.}
\label{fig:OTF}
\end{center}
\end{figure}

The optical transfer function, OTF, is defined as the Fourier transform of the impulse response, see Figure \ref{fig:OTF}. Using the Fourier transform properties it can also be given by the autocorrelation of the pupil function,

\begin{equation}
\overset{\sim}{h}_{optical}(\xi,\eta) = \mathcal{F}\left\lbrace h_{optical}(x,y)\right\rbrace = (P \otimes P)(\xi,\eta),
\label{eqt:OTF}
\end{equation}

where $\xi$ and $\eta$ are the conjugate variables of $x$ and $y$ with respect to the Fourier transform.

\subsection{From Object to Image}
\label{subsec:FromOtoI}

A detector only senses the energy distribution produced by an electromagnetic wave. Therefore, an image is given by the square modulus of the complex amplitude at the focal plane,

\begin{equation}
i(x,y) = |U(x,y)|^2 = |\iint_{-\infty}^{\infty} h(x,y;u,v)U(u,v)dudv|^2.
\label{eqt:IsqAmplitude}
\end{equation}

This integral simplifies differently depending on the type of object we are observing. For a coherent object, \citet[Chapter 6.2]{goodman_1968} showed that the imaging is linear in complex amplitude, thus eqt. \eqref{eqt:IsqAmplitude} becomes,

\begin{equation}
i(x,y) = |\iint_{-\infty}^{\infty} h(x-\overset{\sim}{u},y-\overset{\sim}{v})U(\overset{\sim}{u},\overset{\sim}{v})d\overset{\sim}{u}d\overset{\sim}{v}|^2,
\label{eqt:convolution_hUo}
\end{equation}

where $(\overset{\sim}{u} = Mu,\overset{\sim}{v}= Mv)$ are the normalized object coordinates and $M$ is the magnification of the imaging system. The image is given by the squared modulus of the convolution of the impulse response and the object complex amplitude.

For an incoherent object,  \citet[Chapter 6.2]{goodman_1968} showed that the imaging is linear in intensity,

\begin{equation}
i(x,y) = \iint_{-\infty}^{\infty}|h(x-\overset{\sim}{u},y-\overset{\sim}{v})|^2o(\overset{\sim}{u},\overset{\sim}{v})d\overset{\sim}{u}d\overset{\sim}{v} = (h_{optical}\otimes o)(x,y),
\label{eqt:imageObjectrel}
\end{equation}

where $o(x,y) = |U(x,y)|^2$. We can recognize the convolution of an object with an intensity impulse response, $h_{optical}(x,y) = |h(x,y)|^2$. 

The image that an optical system gives of a point source is called the point spread function, PSF, or impulse response, IR, of the system, see Figure \ref{fig:PSF}. A point source is characterized by an infinite distance to the instrument and therefore the wave is planar, which means that $U(\xi,\eta) = P(\xi,\eta)$. The PSF or IR is given by,

\begin{equation}
h_{optical}(x,y) = |\left[\mathcal{F}\left\lbrace P(\xi,\eta) \right\rbrace\right](x,y)|^2
\label{eqt:impulseResponse}
\end{equation}

The domain where the impulse response is invariant under translation is called the \textbf{isoplanatic domain}.



\section{Aberrations}
\label{sec:Aberrations}

\subsection{Sources of aberrations}
\label{subsec:SourcesAb}

\subsection{Zernike polynomials}
\label{subsec:ZernikePol}
\chapter{Phase Diversity Experiment} 

\label{PDExp}

%----------------------------------------------------------------------------------------
%	SECTION 1
%----------------------------------------------------------------------------------------

\section{Theoretical Background}



%----------------------------------------------------------------------------------------
%	SECTION 2
%----------------------------------------------------------------------------------------

\section{Experimental Setup}

The design of the experiment was already done by \citet{Bouxin_PDM}.  The setup is built according to her plans and specificationsS.

The experiment is mounted on a pressurized legs optical table. The setup contains six components : a light source, an entrance pupil, an imaging system, a converging lens to focus the beam on the camera, a camera and a wavefront sensor.

\subsection{Light source}

\begin{wrapfigure}{r}{0.4\textwidth}
\centering
\includegraphics[width=0.4\textwidth]{Figures/WFdistantSource.PNG}
\decoRulewrapFig
\caption[Wavefront curvature]{Wavefront curvature for different source's distances, z. r represents the characteristic size of the arc of interest.}
\label{fig:WFdistantSource}
\end{wrapfigure}

The final application of the phase diversity will be to characterize the optical aberrations induced by the imperfect optical path to a scientific detector of a telescope. For this reason, the light source has to simulate a distant star aberration-free wavefront. A distant star wavefront is considered planar since the object distance, z, is far greater than the telescope size, r, see Fig. \ref{fig:WFdistantSource}. The source of our experiment must then be characterized by planar wavefront.

In order to obtain such a planar wavefront at the entrance pupil, the light source consist of a "pigtailed laser diode", a f=11mm converging lens, a pinhole and a f=200mm converging lens. The pigtailed laser diode, LPS-635-FC \citet{pigtailedLaserDiode}, emits a Gaussian beam centred at 637.5 nm slightly diverging. The converging lens concentrates the beam at the center of the 10$\mu$m pinhole to filter the noise. The second converging lens collimates the beam, obtaining a collimated beam with a planar wavefront.

\begin{figure}
\centering
    \begin{subfigure}{0.5\textwidth}
        \includegraphics[width=\textwidth]{Figures/source.png}
        \caption{Source ray tracing.}
        \label{fig:sourceRayTracing}
    \end{subfigure}
    \quad
    \begin{subfigure}{0.3\textwidth}
        \includegraphics[width=\textwidth]{Figures/pinholeEffect.png}
        \caption{Beam view before and after the pinhole.}
        \label{fig:pinholeEffect}
    \end{subfigure}
    \caption{Source schema and pinhole effect on the beam.}
\end{figure}

\subsection{Entrance pupil}

The entrance pupil of our optical system is a circular aperture of 3.2mm diameter placed after the collimating lens of the light source. 

%------------------------------------------------------------------------------
%	SECTION 3
%----------------------------------------------------------------------------

\section{Results}

This section presents the results of the phase diversity experiment, with the introduction of different sources of aberration.

\subsection{Astigmatism}

The first aberration studied in this work is the astigmatism aberration introduced by a tilted parallel plane plate (link to section). A parallel plane plate introduces astigmatism in addition to the defocus introduced by a plate perpendicular to the optical axis. The astigmatism is due to the fact that symmetric rays with respect to the optical axis have an optical path difference. 

\chapter{Analytical phase diversity algorithm} 
\label{ch:ourPD}

After having tested and used the ONERA algorithm, we can pass to the main goal of this work, the development of our phase diversity algorithm. In this chapter, we will describe the algorithm and its implementation. And we will present the result of its testing with simulated PSFs. We will present also an approach, that could be used on a telescope equipped with a deformable mirror, increasing the validity domain of the method. And finally we will compare our algorithm with the ONERA algorithm.

\section{Analytical algorithm}
\label{sec:AnAlgo}

\subsection{Algorithm description}
\label{subsec:ANalgoDesc}

This algorithm uses an analytical approach, developed by Dr. Laurent Jolissaint, to retrieve the phase of the wavefront \textbf{induced by a known point source object}. We assume the object known, because we want to correct for the static aberrations present in the optical system to the scientific detector and thus we use a point source to illuminate the optical system, either a star or a laser.

As we have seen in section \ref{sec:ImSystem}, the PSF of an optical system correspond to the image it gives of a point source,

\begin{equation}
PSF(x,y) = \frac{1}{S_p^2}|\left[\mathcal{F}\left\lbrace P(\xi,\eta)A(\xi,\eta)e^{-j\phi(\xi,\eta)} \right\rbrace\right](x,y)|^2,
\label{eqt:PSF}
\end{equation}

where $P(\xi,\eta)$ is the exit pupil function, $A(\xi,\eta)$ is the amplitude of the wave through the exit pupil, $\phi(\xi,\eta)$ is the phase of the wavefront and $S_p$ is the exit pupil surface. In the following we will omit the coordinates to simplify the notation. The unit of the PSF is directly the Strehl ratio. Under the assumption that we have weak aberrations, we can expand the exponential term,

\begin{equation}
exp(-j\phi)\approx 1 - j\phi - \frac{\phi^2}{2} + O(\phi^3),
\label{eqt:expansionPhase}
\end{equation}

replacing the exponential by its expansion in eqt.\eqref{eqt:PSF} leads to,

\begin{equation}
S_p^2 PSF \cong |\mathcal{F}\left\lbrace PA (1-j\phi-\frac{\phi^2}{2}) \right\rbrace|^2
\label{eqt:PSFwthPhaseExpand}
\end{equation}

Developing eqt. \eqref{eqt:PSFwthPhaseExpand}, keeping only the terms up to the second order, assuming that the amplitude through the pupil $A(\xi,\eta)$ is constant and unitary since we have a point source object and using the well known complex relations,

\begin{align}
a + a^* &= 2 \Re \lbrace a \rbrace \nonumber \\
a - a^* &= 2j \Im \lbrace a \rbrace, \nonumber
\end{align}

we obtain the following relation,

\begin{equation}
S_p^2 PSF \cong |\widetilde{P}|^2 + |\widetilde{P\phi}|^2 + 2\Im\lbrace \widetilde{P^*}\widetilde{P \phi}\rbrace - 2\Re\lbrace \widetilde{P^*}\widetilde{P \phi^2}\rbrace
\label{eqt:devPSFwthPhaseExpand}
\end{equation}

Defining $\Delta PSF$ as the difference between eqt. \eqref{eqt:devPSFwthPhaseExpand} for an arbitrary optical system and its perfect equivalent, we obtain the following expression,

\begin{equation}
\Delta PSF = S_p^2 PSF - S_p^2 PSF_{perfect} = |\widetilde{P\phi}|^2 + 2\Im\lbrace \widetilde{P^*}\widetilde{P \phi}\rbrace - 2\Re\lbrace \widetilde{P^*}\widetilde{P \phi^2}\rbrace,
\label{eqt:DeltaPSF}
\end{equation}

where $S_p^2 PSF_{perfect}$ is equal for a equivalent perfect system with the same pupil to $|\widetilde{P}|^2$. One can decompose $\phi$ into its even and odd phase, $\psi$ and $\gamma$ respectively,

\begin{equation}
\phi = \psi + \gamma
\label{eqt:Phidecomposed}
\end{equation}

Developing eqt. \eqref{eqt:DeltaPSF} after replacing $\phi$ by its decomposition and using the properties of the Fourier transform of real and purely even or odd functions, we get the following expression,

\begin{equation}
\Delta PSF = |\widetilde{P\psi}|^2 + |\widetilde{P\gamma}|^2 + 2\Im\lbrace \widetilde{P^*}\widetilde{P \gamma}\rbrace - \Re\lbrace \widetilde{P^*}\widetilde{P \psi^2}\rbrace- \Re\lbrace \widetilde{P^*}\widetilde{P \gamma^2}\rbrace
\label{eqt:DeltaPSFdeveloped}
\end{equation}

We can decompose $\Delta PSF$ into its even and odd components,

\begin{align}
\Delta PSF_{even} &= |\widetilde{P\psi}|^2 + |\widetilde{P\gamma}|^2 - \Re\lbrace \widetilde{P^*}\widetilde{P \psi^2}\rbrace- \Re\lbrace \widetilde{P^*}\widetilde{P \gamma^2}\rbrace, \label{eqt:DeltaPSFeven}\\
\Delta PSF_{odd} &= 2\Im\lbrace \widetilde{P^*}\widetilde{P \gamma}\rbrace \label{eqt:DEltaPSFodd},
\end{align}

This equation system shows that we can retrieve the odd part of the phase easily with eqt. \eqref{eqt:DEltaPSFodd}. But eqt. \eqref{eqt:DeltaPSFeven} clearly reveals the indetermination of the phase retrieval with only one image, as the sign of the even part of $\phi$ can not be determine. In order to raise this indetermination, as exposed in section \ref{sec:principle}, we need to introduce a phase diversity $\delta\phi$. We can modify the pupil function $P$ in order to take into account this introduced diversity,

\begin{equation}
P_{\delta} \equiv P e^{-j\delta\phi} = P(cos(\delta\phi)-jsin(\delta\phi)) = P(C-iS) 
\label{eqt:pupilDeltaphi}
\end{equation}

The expression of $\Delta PSF_{\delta\phi}$, which is the $\Delta PSF$ at the defocus plane, is found by replacing $P$ by $P_{\delta}$ in eqt. \eqref{eqt:DeltaPSFdeveloped}, we give directly the expressions of the even and odd components by taking into account that the phase is only define on the pupil ($P\phi=\phi$) to simplify the reading,

\begin{align}
\Delta PSF_{\delta\phi, even} &= |\widetilde{C\psi}|^2 + |\widetilde{C\gamma}|^2 +|\widetilde{S\psi}|^2 + |\widetilde{S\gamma}|^2 -2\widetilde{PC}^*\widetilde{S\psi}+2\widetilde{PS}^*\widetilde{C\psi} \nonumber\\
&-\widetilde{PC}^*\widetilde{C\psi^2}-\widetilde{PC}^*\widetilde{C\gamma^2}-\widetilde{PS}^*\widetilde{S\psi^2}-\widetilde{PS}^*\widetilde{S\gamma^2} \label{eqt:DeltaPSFevenDef}\\
\Delta PSF_{\delta\phi, odd} &= 2\widetilde{C\psi}^*\Im\lbrace\widetilde{S\gamma}\rbrace+2\Im\lbrace\widetilde{C\gamma}^*\rbrace\widetilde{S\psi}+2\widetilde{PC}^*\Im\lbrace\widetilde{C\gamma}\rbrace+2\widetilde{PS}^*\Im\lbrace\widetilde{S\gamma}\rbrace \nonumber\\
&+2j\widetilde{PC}^*\widetilde{S\psi\gamma}-2j\widetilde{PS}^*\widetilde{C\psi\gamma}.\label{eqt:DeltaPSFoddDef}
\end{align}

Eqt. \eqref{eqt:DEltaPSFodd}, eqt. \eqref{eqt:DeltaPSFevenDef} and eqt. \eqref{eqt:DeltaPSFoddDef} allow to retrieve the complete phase of the optical system under the assumption of weak aberrations. The retrieval numerical method uses the decomposition of the even and odd part of the phase on the Zernike polynomials,

\begin{align}
\psi &= \sum\limits_{js\ even} a_j Z_j \label{eqt:evenPhaseDecomp}\\
\gamma &= \sum\limits_{js\ odd} a_j Z_j. \label{eqt:oddPhaseDecomp}
\end{align}

This allows to have a linear system of equations with respect to the Zernike coefficient $a_j$ using eqts. \eqref{eqt:DEltaPSFodd} and \eqref{eqt:DeltaPSFoddDef}, but a problem arises as we tried to retrieve the phase of a purely even phase. The equations gave us an odd phase part equals to zero but also an even phase part equal to zero. This comes from the fact that in eqt. \eqref{eqt:DeltaPSFoddDef}, each term is multiplied by the odd phase component. And we could not use eqts. \eqref{eqt:DeltaPSFeven} or \eqref{eqt:DeltaPSFevenDef}, due to the squared modulus of the even and odd phase component, which rendered our system of of equation non-linear.

One way to get around this issue is to add another diversity, which is equal in amplitude to the first one but its inverse. So we have two diversity given by,

\begin{equation}
\delta\phi_+ = \delta\phi \mathrm{\ and \ } \delta\phi_- = -\delta\phi
\label{eqt:diversities}
\end{equation}

Using the new diversity, eqt. \eqref{eqt:DEltaPSFodd} allows to determine the odd part of the phase as before, but now for the even part of the phase we compute the difference between $\Delta PSF_{\delta\phi_+,even}$ and $\Delta PSF_{\delta\phi_-,even}$. This gives us the following system of equation,

\begin{align}
\Delta PSF_{odd} &= 2\Im\lbrace \widetilde{P^*}\widetilde{P \gamma}\rbrace \\
\Delta PSF_{\delta\phi_+, even}-\Delta PSF_{\delta\phi_-, even} &= -4\widetilde{PC}^*\widetilde{S\psi} +4\widetilde{PS}^*\widetilde{C\psi},
\end{align}

which we can use to determine the complete phase of the wavefront. We can rewrite it using the decomposition of $\psi$ and $\gamma$ on the Zernike basis,

\begin{align}
\Delta PSF_{odd} &= \sum\limits_{js\ odd} a_j 2\Im\lbrace \widetilde{P^*}\widetilde{P Z_j}\rbrace \label{eqt:DeltaPSFoddonZernike}\\
\Delta PSF_{\delta\phi_+, even}-\Delta PSF_{\delta\phi_-, even} &= \sum\limits_{js\ even} a_j \lbrace-4\widetilde{PC}^*\widetilde{SZ_j} +4\widetilde{PS}^*\widetilde{CZ_j}\rbrace.\label{eqt:DeltaPSF+-DeltaPSF-evenonZernike}
\end{align}

To solve these two equations and find the $a_j$'s, we use a linear regression method. We can rewrite the equations under a vectorial form to clarify the equations by flattening the images,

\begin{align}
\overrightarrow{\Delta PSF}_{odd} &= \underline{Z}_f\vec{a}_{odd}  \label{eqt:DeltaPSFoddonZernikeMAtrix}\\
\overrightarrow{\Delta PSF}_{\delta\phi_+, even}-\overrightarrow{\Delta PSF}_{\delta\phi_-, even} &= \underline{Z}_d \vec{a}_{even}.\label{eqt:DeltaPSF+-DeltaPSF-evenonZernikeMatrix}
\end{align}

where $\underline{Z}_f$ is the $N^2 \mathrm{x} k_{odd}$ matrix regrouping all the terms of $2\Im\lbrace \widetilde{P^*}\widetilde{P Z_j}\rbrace$, $k_{odd}$ is the number of odd Zernike polynomials between $j_{min}$ and $j_{max}$, and $\underline{Z}_d$ is the $N^2 \mathrm{x} k_{even}$ matrix regrouping all the terms of $\lbrace-4\widetilde{PC}^*\widetilde{SZ_j} +4\widetilde{PS}^*\widetilde{CZ_j}\rbrace$, $k_{even}$ is the number of even Zernike polynomials between $j_{min}$ and $j_{max}$.

\subsection{Implementation}

The resolution of the equations as said above is done with a linear regression. The code is structured as following :

\begin{description}
\item[The Class phaseDiversity3PSFs] is the main class of the program, see Appendix \ref{subapp:phaseDiversity3PSFs}. It takes as inputs : the 3 PSFs needed to retrieve the phase as exposed above (inFoc, outFocpos and outFocneg), the displacement of the detector with respect to its focused position ($\Delta z$),

\begin{equation}
\Delta z = \frac{4\Delta\phi}{\pi}\lambda \left(\frac{D}{F}\right)^2,
\label{eqt:Deltaz}
\end{equation}

where $\Delta\phi$ is the Peak To Valley (P2V) dephasing that we want to introduce in the defocused image in the following it will be equal to $2\pi$, $D$ is the pupil diameter and $F$ the focal length of the optical system. It takes also as inputs the wavelength of the incoming light, the pixel size of the detector, the focal length of the optical system, the radius of the pupil and the boundary on $jmin<j<jmax$ the Zernike index. 

The instantiation of all the elements of eqts. \eqref{eqt:DeltaPSFoddonZernikeMAtrix} and \eqref{eqt:DeltaPSF+-DeltaPSF-evenonZernikeMatrix} is done by calling the class method \verb!initiateMatrix1()! and \verb!initiateMatrix2()!. These two methods compute the left and right members of the equations by calling methods present in the script \verb!fs.py!.

\item[The script fs] is  gathering functions needed to retrieve the phase of the wavefront, see Appendix \ref{subapp:fs}. $\Delta PSF_{odd}$ and $\Delta PSF_{\delta\phi_+, even}-\Delta PSF_{\delta\phi_-, even}$ are computed by the two methods \verb!y1(params)! and \verb!y2even(params)!. The matrix elements of $\underline{Z}_f$ are computed by the method \verb!f1j(params)! and the elements of $\underline{Z}_d$ are computed by \verb!f2jeven(params)!. Those two functions needs the Zernike polynomial basis which is coded in \verb!zernike.py!, see Appendix \ref{subapp:zernike}.
\end{description}

The linear regression is done after having initiated all the elements of the equations. We use the \verb!numpy.linalg! class, especially its \verb!lstq! function\footnote{Documentation found at \url{https://docs.scipy.org/doc/numpy/reference/generated/numpy.linalg.lstsq.html\#numpy.linalg.lstsq}}. This functions takes a vector $\vec{b}$ corresponding to the flattened $Delta PSF_{odd}$ or $\Delta PSF_{\delta\phi_+, even}-\Delta PSF_{\delta\phi_-, even}$ and a matrix  $\underline{a}$ corresponding $\underline{Z}_f$ or $\underline{Z}_d$. Then it determines the vector $\vec{x}$ corresponding to the vector $\vec{a}$ that minimizes the euclidean 2-Norm $\lVert b-ax \rVert$ by computing the Moore-Penrose pseudo inverse of $\underline{a}$.

\section{Results}
\label{sec:ourPDresult}


 

\chapter*{Conclusion}
\addcontentsline{toc}{chapter}{Conclusion}
\label{Conclusion}

The wavefront reconstruction is at the forefront of the development of astronomical instruments nowadays. Having bigger and bigger telescope will not bring a lot without methods to counter and correct the effect of the atmosphere. Both the VLT and the ELT as well as the DAG telescope use and will use wavefront sensing and reconstruction to correct the aberrations present and improve the resolution up to the diffraction limit. At the moment phase diversity is still not widely use but its advantages are certain regarding the NCPA and its principle simplicity.

\vspace{1cm}

In our work, we conducted an experiment to test the phase diversity algorithm developed at ONERA by \citet{mugnier_2006}. The results are mixed. The test of the algorithm regarding noise level and $j_{max}$ are interesting. We discovered that, as expected, the noise level in the PSFs has its importance especially in order to reconstruct wavefronts. The zonal method shows that under a noise level of $\sim0.00009$, the high spatial frequencies of the noise are smoothed out and cancelled. For the values of the Zernike coefficient retrieved, the spread is generally small and the noise level is not as critical as to reconstruct the wavefront.

We then computed a Zemax model of our optical system to introduce calibrated astigmatism with a parallel plane plate in Plexiglas. The results are not as expected. The astigmatism retrieved by the phase diversity follows the right behaviour, but there is an offset. Trying to explain it we compared the vertical coma retrieved with the Zemax model, and once again the results were off. We concluded that the most plausible explanation is either that there is an alignment problem or that there is a problem with the usage of the ONERA phase diversity.

The comparison between the Shack-Hartmann and the phase diversity is also not successful. The two wavefront retrieval method shows aberrations of the same amplitude but we have more than 25 nm of difference between the two, which is significant. And we can note conclude anything about the phase diversity as we are unsure about the Shack-Hartmann results after comparing them with the Zemax astigmatism model.

In order to improve the experiment, we would need to take a step back and be sure that we have a way to characterize the system aberrations before trying to use the phase diversity. This means that in the future, we would need to be sure of the Shack-Hartmann results.

\vspace{1cm}

After having tested the ONERA algorithm, the aim was to develop our own phase diversity. It works as expected. And the behaviour with respect to noise and $j_{max}$ are satisfying. An interesting result is the overestimation of the Zernike coefficients when the number of Zernike used by the retrieval is smaller than the one used to construct the simulated PSFs. This overestimation comes from the fact that the algorithm counter-balance for the lost of available components as there is a definite budget of aberrations to spread over a smaller number of Zernike. The validity domain of the method is limited as the method is based on a strong approximation. The method is usable up to a wavefront RMS error of 50 nm. Above the difference becomes too important to be trustworthy. 

A recursive approach has been tested in order to increase the validity domain of the algorithm. This method could be used on a system having a deformable mirror. Th results are really satisfying and we succeeded to multiply by three the validity domain the analytical phase diversity. This is similar to the validity domain of the ONERA algorithm.

Now, before implementing this algorithm on any telescope, we would need to first test it on real PSFs and characterize it as we did for the ONERA algorithm. This could be done after having improved the phase diversity experiment as said above.

\vspace{1cm}

In conclusion, this work was really interesting and I learned the hard way that in science research most of the time it does not go as planned. It was really amazing to do experimental work, as it was the first time I worked as much in the lab. The development of the phase diversity algorithm was a bit more common. But I learned a lot about discrete Fourier transform and their specificity. It has been a great experience. There is still plenty to do especially improving the laboratory experiment in order to correctly test the ONERA algorithm and especially test our own phase diversity on real data.  

%----------------------------------------------------------------------------------------
%	THESIS CONTENT - APPENDICES
%----------------------------------------------------------------------------------------

\appendix



\chapter{Python Code}
\label{AppPythonCode}

\section{Phase Diversity analytical algorithm code}
\label{app:phaseDiversityanAlgoCode}

\subsection{phaseDiversity3PSFs.py}
\label{subapp:phaseDiversity3PSFs}

\begin{lstlisting}
#Class phaseDiversity object to retrieve the phase in the pupil from two psfs in/out of focus

import numpy as np
import fs
import myExceptions
import PSF as psf

class phaseDiversity3PSFs(object):

    def __init__(self,inFoc,outFocpos,outFocneg,deltaZ,lbda,pxsize,F,pupilRadius,jmin,jmax):
#        
#        input:
#        inFoc,outFocpos,outFocneg are the 3 squared PSFs data, (focused, defocused positiv and defocused negative)
#        deltaZ is the displacement of the detector to acquire the two defocused PSFs
#        lbda is the wavelength of the incoming light
#        pxsize is the pixel size of the detector
#        F is the focal length of the imaging system
#        pupilRadius is the radius of the exit pupil
#        jmin and jmax gives the boundary on the js to retrieve
        
        print 'phaseDiversity ...'        
        
        #PSF
        self.inFoc = inFoc
        self.outFocpos = outFocpos
        self.outFocneg = outFocneg
        shapeinFoc = np.shape(self.inFoc)
        shapeoutFocpos = np.shape(self.outFocpos)
        shapeoutFocneg = np.shape(self.outFocneg)
        if shapeinFoc == shapeoutFocpos and shapeinFoc == shapeoutFocneg:
            self.shape = shapeinFoc
        else:
            raise myExceptions.PSFssizeError('the shape of the in/out PSFs is not the same',[shapeinFoc,shapeoutFocpos])
        if shapeinFoc[0]==shapeinFoc[1] and np.mod(shapeinFoc[0],2)==0:
            self.N = shapeinFoc[0]
        else:
            raise myExceptions.PSFssizeError('Either PSF is not square or mod(N,2) != 0',shapeinFoc)
            
        # properties   
        self.deltaZ = deltaZ
        self.lbda = lbda
        self.pxsize = pxsize
        self.F = F
        self.pupilRadius = pupilRadius
        self.dxp = self.F*self.lbda/(self.N*self.pxsize)
        self.rad = int(np.ceil(self.pupilRadius/self.dxp))
        if 2*self.rad > self.N/2.:
            raise myExceptions.PupilSizeError('Npupil (2*rad) is bigger than N/2 which is not correct for the fft computation',[])
        self.NyquistCriterion()
        self.jmin = jmin
        self.jmax = jmax
        self.oddjs = fs.getOddJs(self.jmin,self.jmax)
        self.evenjs = fs.getEvenJs(self.jmin,self.jmax)
        
        # result computation
        self.result = self.retrievePhase()

    def NyquistCriterion(self):
        deltaXfnyq = 0.5*self.lbda/(2*self.pupilRadius)
        deltaXf = self.pxsize/self.F
        if deltaXfnyq < deltaXf : raise myExceptions.NyquistError('the system properties do not respect the nyquist criterion',[])

    def retrievePhase(self):
        y1,A1 = self.initiateMatrix1()
        results1 = np.linalg.lstsq(A1,y1)
        ajsodd = results1[0]
        A1tA1inv= np.linalg.inv(np.matmul(np.transpose(A1),A1))
        n_p = y1.size-1
        ste1 = np.sqrt(results1[1]/n_p * np.diagonal(A1tA1inv))
        
        deltaphi = fs.deltaPhi(self.N,self.deltaZ,self.F,2*self.pupilRadius,self.lbda,self.dxp)
        y2,A2 = self.initiateMatrix2(deltaphi)
        results2 = np.linalg.lstsq(A2,y2)
        ajseven = results2[0]
        A2tA2inv= np.linalg.inv(np.matmul(np.transpose(A2),A2))
        n_p = y2.size-1
        ste2 = np.sqrt(results2[1]/n_p * np.diagonal(A2tA2inv))

        js = np.append(self.oddjs,self.evenjs)
        ajs = np.append(ajsodd,ajseven)
        ajsSte = np.append(ste1,ste2)

        Ixjs = np.argsort(js)
        result = {'js': js[Ixjs], 'ajs': ajs[Ixjs], 'ajsSte': ajsSte[Ixjs]} #,'wavefront':phase}
        return result

    def initiateMatrix1(self):
        deltaPSFinFoc = self.CMPTEdeltaPSF()
        y1 = fs.y1(deltaPSFinFoc)
        A1 = np.zeros((self.N**2,len(self.oddjs)))
        for ij in np.arange(len(self.oddjs)):
            phiJ = fs.f1j(self.oddjs[ij],self.N,self.dxp,self.pupilRadius)
            A1[:,ij] = phiJ
        return y1,A1
    
    def initiateMatrix2(self,deltaphi):
        deltaPSFoutFocpos,deltaPSFoutFocneg = self.CMPTEdeltaPSF(self.deltaZ)
        y2 = fs.y2even(fs.getEvenPart(deltaPSFoutFocpos),fs.getEvenPart(deltaPSFoutFocneg))
        A2 = np.zeros((self.N**2,len(self.evenjs)))
        for ij in np.arange(len(self.evenjs)):
            phiJ = fs.f2jeven(self.evenjs[ij],self.N,deltaphi,self.dxp,self.pupilRadius)
            A2[:,ij] = phiJ
        return y2,A2

    def CMPTEdeltaPSF(self,deltaZ=[]):
        if not deltaZ:
            PSF = psf.PSF([1],[0],self.N,self.dxp,self.pupilRadius)
            return PSF.Sp**2*self.inFoc - PSF.Sp**2*PSF.PSF
        else:
            P2Vdephasing = np.pi*self.deltaZ/self.lbda*(2*self.pupilRadius/self.F)**2/4.
            a4 = P2Vdephasing/2./np.sqrt(3)
            PSF = psf.PSF([4],[a4],self.N,self.dxp,self.pupilRadius)
            return [PSF.Sp**2*self.outFocpos - PSF.Sp**2*PSF.PSF,PSF.Sp**2*self.outFocneg - PSF.Sp**2*PSF.PSF]
\end{lstlisting}

\subsection{fs.py}
\label{subapp:fs}
\begin{lstlisting}
#functions to compute the matrix element of the system of equations Ax = b
import zernike as Z
import phasor as ph
import numpy as np

def f1j(j,N,dxp,pupilRadius): # 1: phij's of matrix A to find a_j odd
    Zj = Z.calc_zern_j(j,N,dxp,pupilRadius)
    FFTZj = scaledfft2(Zj,dxp)

    Lp = N*dxp
    xp = np.arange(-Lp/2,Lp/2,dxp)
    yp = xp

    [Xp,Yp]=np.meshgrid(xp,yp)
    
    pupil = np.float64(np.sqrt(Xp**2+Yp**2)<=pupilRadius)
    FFTPupil = scaledfft2(pupil,dxp)

    return np.ravel(2 * np.real(FFTPupil) * np.imag(FFTZj))

def f2j(j,N,jsodd,ajsodd,deltaphi,dxp,pupilRadius): # 2: phij's of matrix A to find a_j even
    #Get the jth zernike polynomials values on a circular pupil of radius rad
    Zj = Z.calc_zern_j(j,N,dxp,pupilRadius)

    #compute the different 2Dfft given in the equations of deltaPSF
    cosZj = np.cos(deltaphi)*Zj
    sinZj = np.sin(deltaphi)*Zj
    FFTcosZj =  scaledfft2(cosZj,dxp)
    FFTsinZj =  scaledfft2(sinZj,dxp)

    #odd phase
    oddPhasor = ph.phasor(jsodd,ajsodd,N,dxp,pupilRadius)
    oddPhase = oddPhasor.phase
    pupil = oddPhasor.pupil
    cosOddPhase = pupil*np.cos(deltaphi)*oddPhase
    sinOddPhase = pupil*np.sin(deltaphi)*oddPhase
    FFTcosOddPhase =  scaledfft2(cosOddPhase,dxp)
    FFTsinOddPhase =  scaledfft2(sinOddPhase,dxp)

    #2Dfft of pupil function times sin(deltaPhi) and cos(deltaPhi)
    pupilSin = pupil*np.sin(deltaphi)
    pupilCos = pupil*np.cos(deltaphi)
    FFTPupilSin =  scaledfft2(pupilSin,dxp)
    FFTPupilCos =  scaledfft2(pupilCos,dxp)

    FFTsinZjOddPhase = scaledfft2(sinZj*oddPhase,dxp)
    FFTcosZjOddPhase = scaledfft2(cosZj*oddPhase,dxp)

    return np.ravel(2*np.imag(np.conj(FFTcosZj) * FFTsinOddPhase + FFTsinZj * np.conj(FFTcosOddPhase)
            - np.conj(FFTPupilCos) * FFTsinZjOddPhase + np.conj(FFTPupilSin) * FFTcosZjOddPhase))
            
def f2jeven(j,N,deltaphi,dxp,pupilRadius): # 2: phij's of matrix A to find a_j even
    #Get the jth zernike polynomials values on a circular pupil of radius rad
    Zj = Z.calc_zern_j(j,N,dxp,pupilRadius)
    Phasor = ph.phasor([1],[0],N,dxp,pupilRadius)
    pupil = Phasor.pupil
    #compute the different 2Dfft given in the equations of deltaPSF
    cosZj = pupil*np.cos(deltaphi)*Zj
    sinZj = pupil*np.sin(deltaphi)*Zj
    FFTcosZj =  scaledfft2(cosZj,dxp)
    FFTsinZj =  scaledfft2(sinZj,dxp)
    #2Dfft of pupil function times sin(deltaPhi) and cos(deltaPhi)
    pupilSin = pupil*np.sin(deltaphi)
    pupilCos = pupil*np.cos(deltaphi)
    FFTPupilSin =  scaledfft2(pupilSin,dxp)
    FFTPupilCos =  scaledfft2(pupilCos,dxp)

    return np.ravel(-4*np.real(np.conj(FFTPupilCos)*FFTsinZj-np.conj(FFTPupilSin)*FFTcosZj))

def y1(deltaPSFinFoc): #1: yi's of y to find a_j odd
    #compute the odd part of delta PSF
    oddDeltaPSF = getOddPart(deltaPSFinFoc)
    return np.ravel(oddDeltaPSF)

def y2(deltaPSFoutFoc,N,jsodd,ajsodd,deltaphi,dxp,pupilRadius): # 2: yi's of y to find a_j even
    oddDeltaPSF = getOddPart(deltaPSFoutFoc)
    oddPhasor = ph.phasor(jsodd,ajsodd,N,dxp,pupilRadius)
    oddPhase = oddPhasor.phase

    pupilSin = oddPhasor.pupil*np.sin(deltaphi)
    pupilCos = oddPhasor.pupil*np.cos(deltaphi)
    FFTPupilSin =  scaledfft2(pupilSin,dxp)
    FFTPupilCos =  scaledfft2(pupilCos,dxp)
    cosOddPhase = oddPhasor.pupil*np.cos(deltaphi)*oddPhase
    sinOddPhase = oddPhasor.pupil*np.sin(deltaphi)*oddPhase
    FFTcosOddPhase = scaledfft2(cosOddPhase,dxp)
    FFTsinOddPhase =  scaledfft2(sinOddPhase,dxp)

    return np.ravel(oddDeltaPSF - 2*np.real(np.conj(FFTPupilCos))*np.imag(FFTcosOddPhase)
            - 2*np.real(np.conj(FFTPupilSin))*np.imag(FFTsinOddPhase))
            
def y2even(deltaPSFoutFocpos,deltaPSFoutfocneg): #3: yi's of y to find a_j even
    return np.ravel(deltaPSFoutFocpos-deltaPSFoutfocneg)
    
#Other functions----------------------------------------------------------------------
def flipMatrix(M):
    #flip 2D matrix along x and y
    dimM = (np.shape(M))[0]
    Mflipped = np.flipud(np.fliplr(M))
    if np.mod(dimM,2)==0:
        Mflipped = np.roll(Mflipped,1,axis=0)
        Mflipped = np.roll(Mflipped,1,axis=1)
    return Mflipped
def getOddPart(F):
    oddF = (F - flipMatrix(F))/2.
    return oddF
def getEvenPart(F):
    evenF = (F + flipMatrix(F))/2.
    return evenF
def getOddJs(jmin,jmax):
    js = []
    for j in np.arange(jmin,jmax+1):
        [n,m] = Z.noll_to_zern(j)
        if np.mod(m,2) == 1:
            js.append(j)
        else:
            continue
    return np.array(js)
def getEvenJs(jmin,jmax):
    js = []
    for j in np.arange(jmin,jmax+1):
        [n,m] = Z.noll_to_zern(j)
        if np.mod(m,2) == 0:
            js.append(j)
        else:
            continue
    return np.array(js)
def deltaPhi(N,deltaZ,F,D,wavelength,dxp):
    Zj = Z.calc_zern_j(4,N,dxp,D/2.)
    P2Vdephasing = np.pi*deltaZ/wavelength*(D/F)**2/4.
    a4defocus = P2Vdephasing/2/np.sqrt(3)
    return a4defocus*Zj
def cleanZeros(A,threshold):
    A[np.abs(A) < threshold] = 0.
    return A
def scaledfft2(f,dxp):
    return np.fft.ifftshift(np.fft.fft2(np.fft.fftshift(f)))*dxp**2
def RMSE(estimator,target):
    return np.sqrt(np.mean((estimator-target)**2))
def BIAS(estimator,target):
    return np.mean((estimator-target))
    
def RMSwavefrontError(js,ajs):
    if 1 in js:
        return np.sqrt(np.sum(ajs**2)-ajs[js==1]**2)
    else:
        return np.sqrt(np.sum(ajs**2))

\end{lstlisting}

\subsection{myExceptions.py}
\label{subapp:myExceptions}

\begin{lstlisting}
class PSFssizeError(ValueError):
   '''Raise when the size of the PSFs are not correct'''
   def __init__(self, message, foo, *args):
       self.message = message
       self.foo = foo
       super(PSFssizeError, self).__init__(message, foo, *args)

class NyquistError(ValueError):
   '''Raise when the PSFs properties do not respect the nyquist criterion'''
   def __init__(self, message, foo, *args):
       self.message = message
       self.foo = foo
       super(NyquistError, self).__init__(message, foo, *args)

class PupilSizeError(ValueError):
    '''Raise when Npupil is bigger than the size of the PSF N'''
    def __init__(self, message, foo, *args):
        self.message = message
        self.foo = foo
        super(PupilSizeError, self).__init__(message, foo, *args)

\end{lstlisting}

\subsection{zernike.py}
\label{subapp:zernike}

\begin{lstlisting}
import numpy as np

def calc_zern_j(j, N, dxp, pupilRadius):

    Lp = N*dxp

    if (j <= 0):
    		return {'modes':[], 'modesmat':[], 'covmat':0, 'covmat_in':0, 'mask':[[0]]}
    if (N <= 0):
    		raise ValueError("N should be > 0")
    if (dxp <= 0 or dxp >= N):
    		raise ValueError("dxp should be > 0 or < N")

    xp = np.arange(-Lp/2,Lp/2,dxp)
    yp = xp
    [Xp,Yp]=np.meshgrid(xp,yp) 
    r = np.sqrt(Xp**2+Yp**2)
    r = r*(r<=pupilRadius)/pupilRadius
    pup = np.float64(np.sqrt(Xp**2+Yp**2)<=pupilRadius)
    theta = np.arctan2(Yp,Xp)

    Zj = zernike(j,r,theta)*pup
    return Zj

def zernike(j,r,theta):
    n,m = noll_to_zern(j)
    nc = (2*(n+1)/(1+(m==0)))**0.5
    #nc = (2*(n+1)/(1+(m==0)))**0.5
    if (m > 0): return nc*zernike_rad(m, n, r) * np.cos(m * theta)
    if (m < 0): return nc*zernike_rad(-m, n, r) * np.sin(-m * theta)
    return nc*zernike_rad(0, n, r)

def zernike_rad(m, n, r):
    if (np.mod(n-m, 2) == 1):
        return r*0.0
    wf = r*0.0
    for k in range((n-m)/2+1):
        wf += r**(n-2.0*k) * (-1.0)**k * fac(n-k) / ( fac(k) * fac( (n+m)/2.0 - k ) * fac( (n-m)/2.0 - k ) )
    return wf

def noll_to_zern(j):   
    j = int(j)
    if (j == 0):
        raise ValueError("Noll indices start at 1, 0 is invalid.")
    n = 0
    j1 = j-1
    while (j1 > n):
        n += 1
        j1 -= n
    m = (-1)**j * ((n % 2) + 2 * int((j1+((n+1)%2)) / 2.0 ))
    return (n, m)
def fac(n):
    if n == 0:
        return 1
    else:
        return n * fac(n-1)

\end{lstlisting}

\subsection{phasor.py}
\label{subapp:phasor}

\begin{lstlisting}
#class phasor
import numpy as np
import zernike as Z

class phasor(object):

    def __init__(self,js=[1],ajs=[0],N=800,dxp=1,pupilRadius = 200):
        self.js = js
        self.ajs = ajs
        self.N = N
        self.dxp = dxp
        self.pupilRadius = pupilRadius
        
        self.pupil = self.constructPupil()
        
        self.phase = self.constructPhase()
        self.phasor = self.pupil*np.exp(-1j*self.phase)

    def constructPhase(self):
        phase = np.zeros((self.N,self.N))
        for ij, j in enumerate(self.js):
            Zj = Z.calc_zern_j(j,self.N,self.dxp,self.pupilRadius)
            phase += self.ajs[ij]*Zj
        return phase
        
    def constructPupil(self):
        Lp = self.N*self.dxp

        xp = np.arange(-Lp/2,Lp/2,self.dxp)
        yp = xp

        [Xp,Yp]=np.meshgrid(xp,yp)
        
        pup = np.float64(np.sqrt(Xp**2+Yp**2)<=self.pupilRadius)
        return pup

\end{lstlisting}

\subsection{PSF.py}
\label{subapp:PSF}

\begin{lstlisting}
import numpy as np
import phasor as ph
import fs

class PSF(object):

    def __init__(self,js=[1],ajs=[0],N=400,dxp=1.,pupilRadius = 67.):
        self.phasor = ph.phasor(js,ajs,N,dxp,pupilRadius) #phasor in the pupil
        self.Sp = np.sum(self.phasor.pupil)*dxp**2 #pupil surface
        self.FFTphasor = fs.scaledfft2(self.phasor.phasor,dxp) #fourier transform of the phasor
        self.PSF = np.abs(self.FFTphasor)**2/self.Sp**2 #PSF on the focal plane
        self.FFTpupil = fs.scaledfft2(self.phasor.pupil,dxp) #fourier transform of the pupil function which gives the perfect PSF
        self.perfectPSF = np.abs(self.FFTpupil)**2/self.Sp**2 #perfect PSF
        self.deltaPSF = self.Sp**2*self.PSF-self.Sp**2*self.perfectPSF #deltaPSF
\end{lstlisting}

\section{Acquisition Code: Ximea Camera}
\label{app:AcquisitionCodeXimea}

\subsection{AlignementScriptXimeaCamera.py}
\label{subapp:AlignementScriptXimeaCamera}

\begin{lstlisting}
##Script to compute the FWHM of the beam on the camera averaging
 #over "nbrImgAveraging" images and see which position minimizes it.

from ximea import xiapi
import numpy as np
from matplotlib import pyplot  as plt
import scipy.optimize as opt
import datetime
import functionsXimea as fX
import seaborn as sns
import os
sns.set()
#%% instanciation

dataFolderPath = '...'
plotFolderPath = '...'
#create the matrix grid of the detector CCD
x = np.linspace(0,1280,1280)
y = np.linspace(0,1024,1024)
x, y = np.meshgrid(x, y)
 
#initial guess for the fit depending on the position of the beam in the CCD
initial_guess = [250, 481, 706, 3, 3] # [max PSF,y,x,sigmay,sigmax]

#number of image to average
nbrImgAveraging = 10

#%%data acquisition and treatment

#create instance for first connected camera
cam = xiapi.Camera()
#start communication
print('Opening camera...')
cam.open_device()
#settings
cam.set_imgdataformat('XI_MONO8') #XIMEA format 8 bits per pixel
cam.set_gain(0)
#create instance of Image to store image data and metadata
img = xiapi.Image()
#start data acquisition
print('Starting data acquisition...')
if cam.get_acquisition_status() == 'XI_OFF':
         cam.start_acquisition()

cam.set_exposure(fX.determineUnsaturatedExposureTime(cam,img,[60,10000],1))

#instanciation for the while loop
answer ='y'
i=0
relativePos = []
data = []
data_fitted = []
FWHMx = []
FWHMy = []
x0 = []
y0 = []
sigmaX0 = []
sigmaY0 = []

while answer == 'y':

    try:
        relativePos.append(float(raw_input('What is the position on the screw [mm] ? ')))
    except ValueError:
        print('Not a float number')
        
    [tmpdata,stdData] = fX.acquireImg(cam,img,nbrImgAveraging)
    data.append(tmpdata)
    #Fit the img data on the 2D Gaussian to compute the FWHM
    print('Fitting 2D Gaussian...')
    popt, pcov = opt.curve_fit(fX.TwoDGaussian, (x,y), data[i].ravel(), p0 = initial_guess)
    print('Fitting done')

    FWHMx.append(2*np.sqrt(2*np.log(2))*popt[3])
    FWHMy.append(2*np.sqrt(2*np.log(2))*popt[4])
    x0.append(popt[2])
    sigmaX0.append(popt[4])
    y0.append(popt[1])
    sigmaY0.append(popt[3])

    print 'Fig %d : (x,y) = (%3.2f,%3.2f), FWHM x = %3.2f, FWHM y  = %3.2f' %(i,x0[i],y0[i],FWHMx[i],FWHMy[i])

    data_fitted.append(fX.TwoDGaussian((x, y), popt[0],popt[1],popt[2],popt[3],popt[4]).reshape(1024, 1280))

    #plot the beamspot
    fig, ax = plt.subplots(1, 1)
    ax.imshow(data[i], cmap=plt.cm.jet,origin='bottom',
        extent=(x.min(), x.max(), y.min(), y.max()))
    ax.contour(x, y, data_fitted[i], 5, colors='w',linewidths=0.8)
    plt.xlim( (popt[2]-4*popt[4], popt[2]+4*popt[4]) )
    plt.ylim( (popt[1]-4*popt[3], popt[1]+4*popt[3]) )
    plt.show()

    #ask if the person wants to acquire a new image to improve the alignement
    pressedkey = raw_input('Do you want to acquire an other image [y (yes) or n (no)]: ')
    if (pressedkey =='n'):
        answer = pressedkey
    #increase i
    i+=1

#stop data acquisition
print('Stopping acquisition...')
cam.stop_acquisition()

#stop communication
cam.close_device()

#convert list to np.array
relativePos = np.array(relativePos)
data = np.array(data)
FWHMx = np.array(FWHMx)
FWHMy = np.array(FWHMy)
x0 = np.array(x0)
y0 = np.array(y0)
sigmaX0 = np.array(sigmaX0)
sigmaY0 = np.array(sigmaY0)

#plot the FWHM vs. relPos
fig, ax = plt.subplots(1,1)
ind = np.argsort(relativePos)
ax.plot(relativePos[ind],(np.sqrt(FWHMx**2+FWHMy**2))[ind])
ax.set_xlabel('Position [mm]')
ax.set_ylabel('FWHM [px]')
ax.grid()
date = datetime.datetime.today()
if not os.path.isdir(plotFolderPath):
    os.makedirs(plotFolderPath)
plt.savefig(plotFolderPath+date.strftime('%Y%m%d%H%M%S')+'FWHM_pos.pdf')
plt.savefig(plotFolderPath+date.strftime('%Y%m%d%H%M%S')+'FWHM_pos.png')


indOfMinFWHM = np.argmin(np.sqrt(FWHMx**2+FWHMy**2))

fig, axarr = plt.subplots(1,np.size(data,0))
#plot all the images besides each other
for iImg in ind:
    axarr[iImg].imshow(data[iImg], cmap=plt.cm.jet,origin='bottom',
        extent=(x.min(), x.max(), y.min(), y.max()))
#    axarr[iImg].contour(x, y, data_fitted[iImg], 5, colors='w',linewidths=0.8)
    axarr[iImg].set_xlim( (x0[iImg]-12, x0[iImg]+12) )
    axarr[iImg].set_ylim( (y0[iImg]-12, y0[iImg]+12) )
    axarr[iImg].set_yticklabels('',visible=False)
    axarr[iImg].set_xticklabels('',visible=False)
    axarr[iImg].set_title('%5.3f mm'%relativePos[iImg],fontsize=8)
    if iImg == indOfMinFWHM:
        axarr[iImg].set_frame_on(True)
        for pos in ['top', 'bottom', 'right', 'left']:
            axarr[iImg].spines[pos].set_edgecolor('r')
            axarr[iImg].spines[pos].set_linewidth(2)
    else:
        axarr[iImg].set_frame_on(False)
plt.show()
date = datetime.datetime.today()

plt.savefig(plotFolderPath+date.strftime('%Y%m%d%H%M%S')+'ImgPSF.pdf')
plt.savefig(plotFolderPath+date.strftime('%Y%m%d%H%M%S')+'ImgPSF.png')


#save data
if not os.path.isdir(dataFolderPath):
    os.makedirs(dataFolderPath)
date = datetime.datetime.today()
np.save(dataFolderPath+date.strftime('%Y%m%d%H%M%S')+'data.npy',data)
np.save(dataFolderPath+date.strftime('%Y%m%d%H%M%S')+'relativePos.npy',relativePos)
\end{lstlisting}

\subsection{AcquisAndSaveXimea.py}
\label{subapp:AcquisAndSaveXimea}

\begin{lstlisting}
#%% Script to acquire images average over nbrImgAveraging images and save them into fits file

from ximea import xiapi
import datetime
import functionsXimea as fX
import winsound
import numpy as np

#%%instanciation --------------------------------------------------------------
#number of image to average
nbrImgAveraging = 5000
numberOfFinalImages = 1

#Cropping information
sizeImg = 256

#Parameter of camera and saving
folderPathCropped = 'data//cropped/20/'
darkFolderPathCropped = '.data/dark//cropped/20/'
folderPathFull = 'data/full/'
darkFolderPathFull = '/data/dark//full/'
nameCamera = 'Ximea'
focusPos = 11.63

#Sound
duration = 1000  # millisecond
freq = 2000  # Hz

#initial guess for the fit depending on the position of the beam in the CCD
initial_guess = [250, 468, 954, 3, 3] # [max PSF,y,x,sigmay,sigmax]

#------------------------------------------------------------------------------
#%% data acquisition ----------------------------------------------------------

#Opening the connection to the camera
cam = xiapi.Camera()
cam.open_device()
cam.set_imgdataformat('XI_MONO8') #XIMEA format 8 bits per pixel
cam.set_gain(0)

img = xiapi.Image()
if cam.get_acquisition_status() == 'XI_OFF':
    cam.start_acquisition()
#%% exposition
cond = 1
while bool(cond):
    source = ''
    winsound.Beep(freq, duration)
    source = int(raw_input('Is the source turned on and at focus point (usually %5.3f mm) (yes = 1) ? '%focusPos))
    if source == 1:
        cond = 0
    else:
        print 'Please turn on the source and place the camera on the focus point (%5.3f mm)'%focusPos

if bool(source):
    #Set exposure time
    cam.set_exposure(fX.determineUnsaturatedExposureTime(cam,img,[1,10000],1))
    #get centroid
    centroid = fX.acquirePSFCentroid(cam,img,initial_guess)
    print 'centroid at (%d, %d)' %(centroid[0],centroid[1])

#%%Acquire images at different camera position

acquire = 1
while bool(acquire):
    cond = 1
    while bool(cond):
        dark = ''
        winsound.Beep(freq, duration)
        dark = int(raw_input('Is the source turned off (yes = 1) ? '))
        if dark == 1:
            cond = 0
        else:
            print 'Please shut down the source.'

    winsound.Beep(freq, duration)
    pos = float(raw_input('What is the position of the camera in mm focused (%5.3f mm) dephase 2Pi (pos+ = %5.3f mm, pos- = %5.3f) ? '%(focusPos,focusPos+3.19,focusPos-3.19)))

    if bool(dark):
        print 'Acquiring dark image...'
        # Acquire dark images
        [darkData,stdDarkData] = fX.acquireImg(cam,img,nbrImgAveraging)
        print 'Cropping'
        [darkdataCropped,stddarkDataCropped] = fX.cropAroundPSF(darkData,stdDarkData,centroid,sizeImg,sizeImg)
        print 'saving'        
        fX.saveImg2Fits(datetime.datetime.today(),darkFolderPathCropped,nameCamera,darkdataCropped,stddarkDataCropped,str(int(np.around(100*(focusPos-pos),0))),nbrImgAveraging)
        fX.saveImg2Fits(datetime.datetime.today(),darkFolderPathFull,nameCamera,darkData,stdDarkData,str(int(np.around(100*(focusPos-pos),0))),nbrImgAveraging)

    #Acquire images -------------------------
    cond = 1
    while bool(cond):
        source = ''
        winsound.Beep(freq, duration)
        source = int(raw_input('Is the source turned on (yes = 1) ? '))
        if source == 1:
            cond = 0
        else:
            print 'Please place turn on the camera'

    if bool(source):
        print 'Acquiring images...'
        # Acquire focused images
        for iImg in range(numberOfFinalImages):
            imgNumber = iImg+1
            print 'Acquiring Image %d'%imgNumber
            [data,stdData] = fX.acquireImg(cam,img,nbrImgAveraging)
            print 'Cropping'
            [dataCropped,stdDataCropped] = fX.cropAndCenterPSF(data-darkData,stdData+stdDarkData,sizeImg,initial_guess)
            print 'Saving'
            fX.saveImg2Fits(datetime.datetime.today(),folderPathCropped,nameCamera,dataCropped,stdDataCropped,str(int(np.around(100*(focusPos-pos),0))),nbrImgAveraging)
            fX.saveImg2Fits(datetime.datetime.today(),folderPathFull,nameCamera,data-darkData,stdData+stdDarkData,str(int(np.around(100*(focusPos-pos),0))),nbrImgAveraging)

    cond = 1
    while bool(cond):
        acquire = ''
        winsound.Beep(freq, duration)
        acquire = int(raw_input('Do you want to acquire at an other camera position (yes = 1, no = 0) ? '))
        if acquire == 1:
            cond = 0
        elif acquire == 0:
            cond = 0
        else:
             print 'please answer with 0 or 1 for no or yes, respectively'


##Stop the acquisition
cam.stop_acquisition()
cam.close_device()

print 'Acquisition finished'

\end{lstlisting}

\subsection{functionsXimea.py}
\label{subapp:functionsXimea}

\begin{lstlisting}
import numpy as np
import pyfits
import os
import scipy.optimize as opt
#%% Functions -----------------------------------------------------------------


#Create and save .fits from numpy array
def saveImg2Fits(date,folderPath,Detector,data,stdData,pos,nbrAveragingImg):

    #date : datetime at which the data where taken
    #folderPath : where to save the data$
    #Detector : name of detector (ex:Ximea)
    #data : np.array containing the image
    #stdData: np.array containing the error on each pixel
    #pos : the position of the camera on the sliding holder in mm

    imgHdu = pyfits.PrimaryHDU(data)
    stdHdu = pyfits.ImageHDU(stdData,name = 'imgStdData')
    hdulist = pyfits.HDUList([imgHdu,stdHdu])

    if not os.path.isdir(folderPath):
        os.makedirs(folderPath)

    hdulist.writeto(folderPath + date.strftime('%Y%m%d%H%M%S')+'_'+Detector+'_'+pos+'.fits')

def acquireImg(cam,img,nbrImgAveraging):
    imgData = np.zeros([1024,1280])
    stdData = np.zeros([1024,1280])
    for iImg in range(nbrImgAveraging) :
        #print iImg
        cam.get_image(img)
        imgTmpData = img.get_image_data_numpy()
        imgData[:,:] += imgTmpData
        stdData += imgTmpData*imgTmpData

    stdData = np.sqrt((stdData-imgData*imgData/nbrImgAveraging)/(nbrImgAveraging-1))/(np.sqrt(nbrImgAveraging))
    imgData = imgData/nbrImgAveraging

    return [imgData,stdData]


def determineUnsaturatedExposureTime(cam,img,exposureLimit,precision):
     exposureTimes = exposureLimit
     if cam.get_acquisition_status() == 'XI_OFF':
         cam.start_acquisition()

     while np.absolute(np.diff(exposureTimes))>precision:
         expTime2check = int(np.round(np.nanmean(exposureTimes)))
         print 'Try expTime : %d [us]\n' %expTime2check
         cam.set_exposure(expTime2check)
         data = acquireImg(cam,img,10)[0]

         if np.sum(data>250)>1:
             exposureTimes[1] = int(np.ceil(np.nanmean(exposureTimes)))
         else:
             exposureTimes[0] = int(np.floor(np.nanmean(exposureTimes)))
         print 'exposure time between %d and %d \n' %(exposureTimes[0],exposureTimes[1])

     return int(np.floor(np.nanmean(exposureTimes)))

def TwoDGaussian((x, y), A, yo, xo, sigma_y, sigma_x):
    g = A*np.exp( - ((x-xo)**2/(2*sigma_x**2) + ((y-yo)**2)/(2*sigma_y**2)))
    return g.ravel()

def acquirePSFCentroid(cam,img,initial_guess):
    #create the matrix grid of the detector CCD

    data = acquireImg(cam,img,200)[0]
    centroid = getPSFCentroid(data,initial_guess)
    return centroid

def getPSFCentroid(data,initial_guess):
    
    x = np.linspace(0,1280,1280)
    y = np.linspace(0,1024,1024)
    x, y = np.meshgrid(x, y)
    print 'fitting'
    popt,pcov = opt.curve_fit(TwoDGaussian, (x,y), data.ravel(), p0 = initial_guess)
    print 'fitting done'
    return [popt[2],popt[1]]


def cropAndCenterPSF(data,stdData,size,initial_guess):
    Xextent = np.size(data,1)
    Yextent = np.size(data,0)

    centroid = getPSFCentroid(data,initial_guess)

    minMarge = np.min([centroid[0],centroid[1],Xextent-centroid[0],Yextent-centroid[1]])

    if minMarge>size/2:
        return cropAroundPSF(data,stdData,centroid,size,size)
    elif minMarge<size/2:
        return cropAroundPSF(data,stdData, centroid,2*minMarge,2*minMarge)


def cropAroundPSF(data,stdData,centroid,sizeX,sizeY):

    pxX = [int(np.floor(centroid[0])-np.ceil(sizeX/2)),int(np.floor(centroid[0])+np.ceil(sizeX/2))]
    pxY = [int(np.floor(centroid[1])-np.ceil(sizeY/2)),int(np.floor(centroid[1])+np.ceil(sizeY/2))]

    dataCropped = data[pxY[0]:pxY[1],pxX[0]:pxX[1]]

    stdDataCropped = stdData[pxY[0]:pxY[1],pxX[0]:pxX[1]]

    return [dataCropped,stdDataCropped]

\end{lstlisting}


\chapter{IDL Code}
\label{AppIDLCode}

\section{Shack-Hartmann Acquisition Code}
\label{app:SHacquisCode}

\subsection{readAndAverageSHdata.pro}
\label{subapp:readAndAverageSHdata}

\begin{lstlisting}
function readAndAverageSHdata, folderPath

fileExt='*.csv'

files = file_search(folderPath+fileExt)

r = readshwfsdata(files[0])

NFiles = n_elements(files)

for ifile = 1,Nfiles-1 do begin
  
  rtmp = readshwfsdata(files[ifile])
  
  r.wavefront = r.wavefront+rtmp.wavefront
  r.zernike[3,*] = r.zernike[3,*]+rtmp.zernike[3,*]
  
endfor

r.wavefront = r.wavefront / NFiles
r.zernike[3,*] = r.zernike[3,*] / Nfiles

return, r
end
\end{lstlisting}

\subsection{readSHWFSdata.pro}
\label{subapp:readSHWFSdata}

\begin{lstlisting}
function readSHWFSdata, filePath

openr, f, filePath, /GET_LUN

iLine = 0
line = ''


coefficient = []
index = []
order = []
frequency = []
wavefront = []

while ~EOF(f) do begin
  readf, f, line
  iLine += 1
  
  ;get the zernike coefficient
  if strmatch(line,'* ZERNIKE FIT *') then begin
    subheaderNbrLines = 5
    for isHd =1,subheaderNbrLines do begin
      readf, f, line
      iLine += 1
    endfor
    
    readf, f, line
    iLine += 1
    sLine = strsplit(line,',',/EXTRACT)
    
    while stregex(sLine[0],'[0-9]+') ne -1 and ~EOF(f) do begin
      index = [[index],[long(sLine[0])]]
      order = [[order],[long(sLine[1])]]
      frequency = [[frequency],[long(sLine[2])]]
      coefficient = [[coefficient],[double(sLine[3])]]
      readf, f, line
      iLine += 1
      sLine = strsplit(line,',',/EXTRACT)
    endwhile
  endif
  
  if strmatch(line,'\*\*\* WAVEFRONT \*\*\*')  then begin
    subheaderNbrLines = 11
    for isHd =1,subheaderNbrLines do begin
      readf, f, line
      iLine += 1
    endfor
    readf, f, line
    iLine += 1
    sLine = strsplit(line,',',/EXTRACT)
    nel = n_elements(sLine)
    while stregex(sLine[0],'[0-9]+') ne -1 and ~EOF(f) do begin
      wavefront = [[wavefront],[double(sLine[1:nel-1])]]
      readf, f, line
      iLine += 1
      sLine = strsplit(line,',',/EXTRACT)
    endwhile
    
  endif
endwhile
free_lun, f

zernike = [index,order,frequency,coefficient]

return, {zernike:zernike,wavefront:wavefront}

end
\end{lstlisting}
% Appendix A

\chapter{Optical Component Datasheets}
\label{AppendixA} 

\section{Pigtailed laser diode}
\label{app:pigtailedLaserDiode}
\includegraphics[width=\textwidth]{../../componentDatasheet/LPS-635-FC-SpecSheet.pdf}
Source : \url{www.thorlabs.com}

\section{Converging lens A220TM-A, f = 11 mm}
\label{app:CL11}
\includegraphics[width=\textwidth]{../../componentDatasheet/A220TM-AutoCADPDF.pdf}
Source : \url{www.thorlabs.com}

\section{Pinhole 10 $\mu$m}
\label{app:pinhole10microns}
\includegraphics[width=\textwidth]{../../componentDatasheet/P10S-AutoCADPDF.pdf}
Source : \url{www.thorlabs.com}

\section{Converging lens AL100200, f = 200 mm}
\label{app:CL200}
\includegraphics[width=\textwidth]{../../componentDatasheet/AL100200-AutoCADPDF.pdf}
Source : \url{www.thorlabs.com}

\section{Converging lens AC254-100-A, f = 100 mm}
\label{app:CL100}
\includegraphics[width=\textwidth]{../../componentDatasheet/AC254-100-A-AutoCADPDF.pdf}
Source : \url{www.thorlabs.com}

\section{Ximea Camera, MQ013MG-E2}
\label{app:ximeaCam}
\begin{center}
\includegraphics[width=0.5\textwidth]{../../componentDatasheet/ximeaPic.png}
\includegraphics[width=0.5\textwidth]{../../componentDatasheet/ximeaSpec.png}
\end{center}
Source : \url{www.ximea.com/en/products/usb3-vision-cameras-xiq-line/mq013mg-e2}

\section{Shack-Hartmann wavefront sensor, WFS150-5C}
\label{app:SHwfs}
\includegraphics[width=\textwidth]{../../componentDatasheet/WFS150-5C.png}
Source : WFS Series Operation Manual, \url{www.thorlabs.com}


%----------------------------------------------------------------------------------------
%	BIBLIOGRAPHY
%----------------------------------------------------------------------------------------

\printbibliography[heading=bibintoc]

%----------------------------------------------------------------------------------------

\end{document}  
