

\chapter{Python Code}
\label{AppPythonCode}

\section{Phase Diversity analytical algorithm code}
\label{app:phaseDiversityanAlgoCode}

\subsection{phaseDiversity3PSFs.py}
\label{subapp:phaseDiversity3PSFs}

\begin{lstlisting}
#Class phaseDiversity object to retrieve the phase in the pupil from two psfs in/out of focus

import numpy as np
import fs
import myExceptions
import PSF as psf

class phaseDiversity3PSFs(object):

    def __init__(self,inFoc,outFocpos,outFocneg,deltaZ,lbda,pxsize,F,pupilRadius,jmin,jmax):
#        
#        input:
#        inFoc,outFocpos,outFocneg are the 3 squared PSFs data, (focused, defocused positiv and defocused negative)
#        deltaZ is the displacement of the detector to acquire the two defocused PSFs
#        lbda is the wavelength of the incoming light
#        pxsize is the pixel size of the detector
#        F is the focal length of the imaging system
#        pupilRadius is the radius of the exit pupil
#        jmin and jmax gives the boundary on the js to retrieve
        
        print 'phaseDiversity ...'        
        
        #PSF
        self.inFoc = inFoc
        self.outFocpos = outFocpos
        self.outFocneg = outFocneg
        shapeinFoc = np.shape(self.inFoc)
        shapeoutFocpos = np.shape(self.outFocpos)
        shapeoutFocneg = np.shape(self.outFocneg)
        if shapeinFoc == shapeoutFocpos and shapeinFoc == shapeoutFocneg:
            self.shape = shapeinFoc
        else:
            raise myExceptions.PSFssizeError('the shape of the in/out PSFs is not the same',[shapeinFoc,shapeoutFocpos])
        if shapeinFoc[0]==shapeinFoc[1] and np.mod(shapeinFoc[0],2)==0:
            self.N = shapeinFoc[0]
        else:
            raise myExceptions.PSFssizeError('Either PSF is not square or mod(N,2) != 0',shapeinFoc)
            
        # properties   
        self.deltaZ = deltaZ
        self.lbda = lbda
        self.pxsize = pxsize
        self.F = F
        self.pupilRadius = pupilRadius
        self.dxp = self.F*self.lbda/(self.N*self.pxsize)
        self.rad = int(np.ceil(self.pupilRadius/self.dxp))
        if 2*self.rad > self.N/2.:
            raise myExceptions.PupilSizeError('Npupil (2*rad) is bigger than N/2 which is not correct for the fft computation',[])
        self.NyquistCriterion()
        self.jmin = jmin
        self.jmax = jmax
        self.oddjs = fs.getOddJs(self.jmin,self.jmax)
        self.evenjs = fs.getEvenJs(self.jmin,self.jmax)
        
        # result computation
        self.result = self.retrievePhase()

    def NyquistCriterion(self):
        deltaXfnyq = 0.5*self.lbda/(2*self.pupilRadius)
        deltaXf = self.pxsize/self.F
        if deltaXfnyq < deltaXf : raise myExceptions.NyquistError('the system properties do not respect the nyquist criterion',[])

    def retrievePhase(self):
        y1,A1 = self.initiateMatrix1()
        results1 = np.linalg.lstsq(A1,y1)
        ajsodd = results1[0]
        A1tA1inv= np.linalg.inv(np.matmul(np.transpose(A1),A1))
        n_p = y1.size-1
        ste1 = np.sqrt(results1[1]/n_p * np.diagonal(A1tA1inv))
        
        deltaphi = fs.deltaPhi(self.N,self.deltaZ,self.F,2*self.pupilRadius,self.lbda,self.dxp)
        y2,A2 = self.initiateMatrix2(deltaphi)
        results2 = np.linalg.lstsq(A2,y2)
        ajseven = results2[0]
        A2tA2inv= np.linalg.inv(np.matmul(np.transpose(A2),A2))
        n_p = y2.size-1
        ste2 = np.sqrt(results2[1]/n_p * np.diagonal(A2tA2inv))

        js = np.append(self.oddjs,self.evenjs)
        ajs = np.append(ajsodd,ajseven)
        ajsSte = np.append(ste1,ste2)

        Ixjs = np.argsort(js)
        result = {'js': js[Ixjs], 'ajs': ajs[Ixjs], 'ajsSte': ajsSte[Ixjs]} #,'wavefront':phase}
        return result

    def initiateMatrix1(self):
        deltaPSFinFoc = self.CMPTEdeltaPSF()
        y1 = fs.y1(deltaPSFinFoc)
        A1 = np.zeros((self.N**2,len(self.oddjs)))
        for ij in np.arange(len(self.oddjs)):
            phiJ = fs.f1j(self.oddjs[ij],self.N,self.dxp,self.pupilRadius)
            A1[:,ij] = phiJ
        return y1,A1
    
    def initiateMatrix2(self,deltaphi):
        deltaPSFoutFocpos,deltaPSFoutFocneg = self.CMPTEdeltaPSF(self.deltaZ)
        y2 = fs.y2even(fs.getEvenPart(deltaPSFoutFocpos),fs.getEvenPart(deltaPSFoutFocneg))
        A2 = np.zeros((self.N**2,len(self.evenjs)))
        for ij in np.arange(len(self.evenjs)):
            phiJ = fs.f2jeven(self.evenjs[ij],self.N,deltaphi,self.dxp,self.pupilRadius)
            A2[:,ij] = phiJ
        return y2,A2

    def CMPTEdeltaPSF(self,deltaZ=[]):
        if not deltaZ:
            PSF = psf.PSF([1],[0],self.N,self.dxp,self.pupilRadius)
            return PSF.Sp**2*self.inFoc - PSF.Sp**2*PSF.PSF
        else:
            P2Vdephasing = np.pi*self.deltaZ/self.lbda*(2*self.pupilRadius/self.F)**2/4.
            a4 = P2Vdephasing/2./np.sqrt(3)
            PSF = psf.PSF([4],[a4],self.N,self.dxp,self.pupilRadius)
            return [PSF.Sp**2*self.outFocpos - PSF.Sp**2*PSF.PSF,PSF.Sp**2*self.outFocneg - PSF.Sp**2*PSF.PSF]
\end{lstlisting}

\subsection{fs.py}
\label{subapp:fs}
\begin{lstlisting}
#functions to compute the matrix element of the system of equations Ax = b
import zernike as Z
import phasor as ph
import numpy as np

def f1j(j,N,dxp,pupilRadius): # 1: phij's of matrix A to find a_j odd
    Zj = Z.calc_zern_j(j,N,dxp,pupilRadius)
    FFTZj = scaledfft2(Zj,dxp)

    Lp = N*dxp
    xp = np.arange(-Lp/2,Lp/2,dxp)
    yp = xp

    [Xp,Yp]=np.meshgrid(xp,yp)
    
    pupil = np.float64(np.sqrt(Xp**2+Yp**2)<=pupilRadius)
    FFTPupil = scaledfft2(pupil,dxp)

    return np.ravel(2 * np.real(FFTPupil) * np.imag(FFTZj))

def f2j(j,N,jsodd,ajsodd,deltaphi,dxp,pupilRadius): # 2: phij's of matrix A to find a_j even
    #Get the jth zernike polynomials values on a circular pupil of radius rad
    Zj = Z.calc_zern_j(j,N,dxp,pupilRadius)

    #compute the different 2Dfft given in the equations of deltaPSF
    cosZj = np.cos(deltaphi)*Zj
    sinZj = np.sin(deltaphi)*Zj
    FFTcosZj =  scaledfft2(cosZj,dxp)
    FFTsinZj =  scaledfft2(sinZj,dxp)

    #odd phase
    oddPhasor = ph.phasor(jsodd,ajsodd,N,dxp,pupilRadius)
    oddPhase = oddPhasor.phase
    pupil = oddPhasor.pupil
    cosOddPhase = pupil*np.cos(deltaphi)*oddPhase
    sinOddPhase = pupil*np.sin(deltaphi)*oddPhase
    FFTcosOddPhase =  scaledfft2(cosOddPhase,dxp)
    FFTsinOddPhase =  scaledfft2(sinOddPhase,dxp)

    #2Dfft of pupil function times sin(deltaPhi) and cos(deltaPhi)
    pupilSin = pupil*np.sin(deltaphi)
    pupilCos = pupil*np.cos(deltaphi)
    FFTPupilSin =  scaledfft2(pupilSin,dxp)
    FFTPupilCos =  scaledfft2(pupilCos,dxp)

    FFTsinZjOddPhase = scaledfft2(sinZj*oddPhase,dxp)
    FFTcosZjOddPhase = scaledfft2(cosZj*oddPhase,dxp)

    return np.ravel(2*np.imag(np.conj(FFTcosZj) * FFTsinOddPhase + FFTsinZj * np.conj(FFTcosOddPhase)
            - np.conj(FFTPupilCos) * FFTsinZjOddPhase + np.conj(FFTPupilSin) * FFTcosZjOddPhase))
            
def f2jeven(j,N,deltaphi,dxp,pupilRadius): # 2: phij's of matrix A to find a_j even
    #Get the jth zernike polynomials values on a circular pupil of radius rad
    Zj = Z.calc_zern_j(j,N,dxp,pupilRadius)
    Phasor = ph.phasor([1],[0],N,dxp,pupilRadius)
    pupil = Phasor.pupil
    #compute the different 2Dfft given in the equations of deltaPSF
    cosZj = pupil*np.cos(deltaphi)*Zj
    sinZj = pupil*np.sin(deltaphi)*Zj
    FFTcosZj =  scaledfft2(cosZj,dxp)
    FFTsinZj =  scaledfft2(sinZj,dxp)
    #2Dfft of pupil function times sin(deltaPhi) and cos(deltaPhi)
    pupilSin = pupil*np.sin(deltaphi)
    pupilCos = pupil*np.cos(deltaphi)
    FFTPupilSin =  scaledfft2(pupilSin,dxp)
    FFTPupilCos =  scaledfft2(pupilCos,dxp)

    return np.ravel(-4*np.real(np.conj(FFTPupilCos)*FFTsinZj-np.conj(FFTPupilSin)*FFTcosZj))

def y1(deltaPSFinFoc): #1: yi's of y to find a_j odd
    #compute the odd part of delta PSF
    oddDeltaPSF = getOddPart(deltaPSFinFoc)
    return np.ravel(oddDeltaPSF)

def y2(deltaPSFoutFoc,N,jsodd,ajsodd,deltaphi,dxp,pupilRadius): # 2: yi's of y to find a_j even
    oddDeltaPSF = getOddPart(deltaPSFoutFoc)
    oddPhasor = ph.phasor(jsodd,ajsodd,N,dxp,pupilRadius)
    oddPhase = oddPhasor.phase

    pupilSin = oddPhasor.pupil*np.sin(deltaphi)
    pupilCos = oddPhasor.pupil*np.cos(deltaphi)
    FFTPupilSin =  scaledfft2(pupilSin,dxp)
    FFTPupilCos =  scaledfft2(pupilCos,dxp)
    cosOddPhase = oddPhasor.pupil*np.cos(deltaphi)*oddPhase
    sinOddPhase = oddPhasor.pupil*np.sin(deltaphi)*oddPhase
    FFTcosOddPhase = scaledfft2(cosOddPhase,dxp)
    FFTsinOddPhase =  scaledfft2(sinOddPhase,dxp)

    return np.ravel(oddDeltaPSF - 2*np.real(np.conj(FFTPupilCos))*np.imag(FFTcosOddPhase)
            - 2*np.real(np.conj(FFTPupilSin))*np.imag(FFTsinOddPhase))
            
def y2even(deltaPSFoutFocpos,deltaPSFoutfocneg): #3: yi's of y to find a_j even
    return np.ravel(deltaPSFoutFocpos-deltaPSFoutfocneg)
    
#Other functions----------------------------------------------------------------------
def flipMatrix(M):
    #flip 2D matrix along x and y
    dimM = (np.shape(M))[0]
    Mflipped = np.flipud(np.fliplr(M))
    if np.mod(dimM,2)==0:
        Mflipped = np.roll(Mflipped,1,axis=0)
        Mflipped = np.roll(Mflipped,1,axis=1)
    return Mflipped
def getOddPart(F):
    oddF = (F - flipMatrix(F))/2.
    return oddF
def getEvenPart(F):
    evenF = (F + flipMatrix(F))/2.
    return evenF
def getOddJs(jmin,jmax):
    js = []
    for j in np.arange(jmin,jmax+1):
        [n,m] = Z.noll_to_zern(j)
        if np.mod(m,2) == 1:
            js.append(j)
        else:
            continue
    return np.array(js)
def getEvenJs(jmin,jmax):
    js = []
    for j in np.arange(jmin,jmax+1):
        [n,m] = Z.noll_to_zern(j)
        if np.mod(m,2) == 0:
            js.append(j)
        else:
            continue
    return np.array(js)
def deltaPhi(N,deltaZ,F,D,wavelength,dxp):
    Zj = Z.calc_zern_j(4,N,dxp,D/2.)
    P2Vdephasing = np.pi*deltaZ/wavelength*(D/F)**2/4.
    a4defocus = P2Vdephasing/2/np.sqrt(3)
    return a4defocus*Zj
def cleanZeros(A,threshold):
    A[np.abs(A) < threshold] = 0.
    return A
def scaledfft2(f,dxp):
    return np.fft.ifftshift(np.fft.fft2(np.fft.fftshift(f)))*dxp**2
def RMSE(estimator,target):
    return np.sqrt(np.mean((estimator-target)**2))
def BIAS(estimator,target):
    return np.mean((estimator-target))
    
def RMSwavefrontError(js,ajs):
    if 1 in js:
        return np.sqrt(np.sum(ajs**2)-ajs[js==1]**2)
    else:
        return np.sqrt(np.sum(ajs**2))

\end{lstlisting}

\subsection{myExceptions.py}
\label{subapp:myExceptions}

\begin{lstlisting}
class PSFssizeError(ValueError):
   '''Raise when the size of the PSFs are not correct'''
   def __init__(self, message, foo, *args):
       self.message = message
       self.foo = foo
       super(PSFssizeError, self).__init__(message, foo, *args)

class NyquistError(ValueError):
   '''Raise when the PSFs properties do not respect the nyquist criterion'''
   def __init__(self, message, foo, *args):
       self.message = message
       self.foo = foo
       super(NyquistError, self).__init__(message, foo, *args)

class PupilSizeError(ValueError):
    '''Raise when Npupil is bigger than the size of the PSF N'''
    def __init__(self, message, foo, *args):
        self.message = message
        self.foo = foo
        super(PupilSizeError, self).__init__(message, foo, *args)

\end{lstlisting}

\subsection{zernike.py}
\label{subapp:zernike}

\begin{lstlisting}
import numpy as np

def calc_zern_j(j, N, dxp, pupilRadius):

    Lp = N*dxp

    if (j <= 0):
    		return {'modes':[], 'modesmat':[], 'covmat':0, 'covmat_in':0, 'mask':[[0]]}
    if (N <= 0):
    		raise ValueError("N should be > 0")
    if (dxp <= 0 or dxp >= N):
    		raise ValueError("dxp should be > 0 or < N")

    xp = np.arange(-Lp/2,Lp/2,dxp)
    yp = xp
    [Xp,Yp]=np.meshgrid(xp,yp) 
    r = np.sqrt(Xp**2+Yp**2)
    r = r*(r<=pupilRadius)/pupilRadius
    pup = np.float64(np.sqrt(Xp**2+Yp**2)<=pupilRadius)
    theta = np.arctan2(Yp,Xp)

    Zj = zernike(j,r,theta)*pup
    return Zj

def zernike(j,r,theta):
    n,m = noll_to_zern(j)
    nc = (2*(n+1)/(1+(m==0)))**0.5
    #nc = (2*(n+1)/(1+(m==0)))**0.5
    if (m > 0): return nc*zernike_rad(m, n, r) * np.cos(m * theta)
    if (m < 0): return nc*zernike_rad(-m, n, r) * np.sin(-m * theta)
    return nc*zernike_rad(0, n, r)

def zernike_rad(m, n, r):
    if (np.mod(n-m, 2) == 1):
        return r*0.0
    wf = r*0.0
    for k in range((n-m)/2+1):
        wf += r**(n-2.0*k) * (-1.0)**k * fac(n-k) / ( fac(k) * fac( (n+m)/2.0 - k ) * fac( (n-m)/2.0 - k ) )
    return wf

def noll_to_zern(j):   
    j = int(j)
    if (j == 0):
        raise ValueError("Noll indices start at 1, 0 is invalid.")
    n = 0
    j1 = j-1
    while (j1 > n):
        n += 1
        j1 -= n
    m = (-1)**j * ((n % 2) + 2 * int((j1+((n+1)%2)) / 2.0 ))
    return (n, m)
def fac(n):
    if n == 0:
        return 1
    else:
        return n * fac(n-1)

\end{lstlisting}

\subsection{phasor.py}
\label{subapp:phasor}

\begin{lstlisting}
#class phasor
import numpy as np
import zernike as Z

class phasor(object):

    def __init__(self,js=[1],ajs=[0],N=800,dxp=1,pupilRadius = 200):
        self.js = js
        self.ajs = ajs
        self.N = N
        self.dxp = dxp
        self.pupilRadius = pupilRadius
        
        self.pupil = self.constructPupil()
        
        self.phase = self.constructPhase()
        self.phasor = self.pupil*np.exp(-1j*self.phase)

    def constructPhase(self):
        phase = np.zeros((self.N,self.N))
        for ij, j in enumerate(self.js):
            Zj = Z.calc_zern_j(j,self.N,self.dxp,self.pupilRadius)
            phase += self.ajs[ij]*Zj
        return phase
        
    def constructPupil(self):
        Lp = self.N*self.dxp

        xp = np.arange(-Lp/2,Lp/2,self.dxp)
        yp = xp

        [Xp,Yp]=np.meshgrid(xp,yp)
        
        pup = np.float64(np.sqrt(Xp**2+Yp**2)<=self.pupilRadius)
        return pup

\end{lstlisting}

\subsection{PSF.py}
\label{subapp:PSF}

\begin{lstlisting}
import numpy as np
import phasor as ph
import fs

class PSF(object):

    def __init__(self,js=[1],ajs=[0],N=400,dxp=1.,pupilRadius = 67.):
        self.phasor = ph.phasor(js,ajs,N,dxp,pupilRadius) #phasor in the pupil
        self.Sp = np.sum(self.phasor.pupil)*dxp**2 #pupil surface
        self.FFTphasor = fs.scaledfft2(self.phasor.phasor,dxp) #fourier transform of the phasor
        self.PSF = np.abs(self.FFTphasor)**2/self.Sp**2 #PSF on the focal plane
        self.FFTpupil = fs.scaledfft2(self.phasor.pupil,dxp) #fourier transform of the pupil function which gives the perfect PSF
        self.perfectPSF = np.abs(self.FFTpupil)**2/self.Sp**2 #perfect PSF
        self.deltaPSF = self.Sp**2*self.PSF-self.Sp**2*self.perfectPSF #deltaPSF
\end{lstlisting}

\section{Acquisition Code: Ximea Camera}
\label{app:AcquisitionCodeXimea}

\subsection{AlignementScriptXimeaCamera.py}
\label{subapp:AlignementScriptXimeaCamera}

\begin{lstlisting}
##Script to compute the FWHM of the beam on the camera averaging
 #over "nbrImgAveraging" images and see which position minimizes it.

from ximea import xiapi
import numpy as np
from matplotlib import pyplot  as plt
import scipy.optimize as opt
import datetime
import functionsXimea as fX
import seaborn as sns
import os
sns.set()
#%% instanciation

dataFolderPath = '...'
plotFolderPath = '...'
#create the matrix grid of the detector CCD
x = np.linspace(0,1280,1280)
y = np.linspace(0,1024,1024)
x, y = np.meshgrid(x, y)
 
#initial guess for the fit depending on the position of the beam in the CCD
initial_guess = [250, 481, 706, 3, 3] # [max PSF,y,x,sigmay,sigmax]

#number of image to average
nbrImgAveraging = 10

#%%data acquisition and treatment

#create instance for first connected camera
cam = xiapi.Camera()
#start communication
print('Opening camera...')
cam.open_device()
#settings
cam.set_imgdataformat('XI_MONO8') #XIMEA format 8 bits per pixel
cam.set_gain(0)
#create instance of Image to store image data and metadata
img = xiapi.Image()
#start data acquisition
print('Starting data acquisition...')
if cam.get_acquisition_status() == 'XI_OFF':
         cam.start_acquisition()

cam.set_exposure(fX.determineUnsaturatedExposureTime(cam,img,[60,10000],1))

#instanciation for the while loop
answer ='y'
i=0
relativePos = []
data = []
data_fitted = []
FWHMx = []
FWHMy = []
x0 = []
y0 = []
sigmaX0 = []
sigmaY0 = []

while answer == 'y':

    try:
        relativePos.append(float(raw_input('What is the position on the screw [mm] ? ')))
    except ValueError:
        print('Not a float number')
        
    [tmpdata,stdData] = fX.acquireImg(cam,img,nbrImgAveraging)
    data.append(tmpdata)
    #Fit the img data on the 2D Gaussian to compute the FWHM
    print('Fitting 2D Gaussian...')
    popt, pcov = opt.curve_fit(fX.TwoDGaussian, (x,y), data[i].ravel(), p0 = initial_guess)
    print('Fitting done')

    FWHMx.append(2*np.sqrt(2*np.log(2))*popt[3])
    FWHMy.append(2*np.sqrt(2*np.log(2))*popt[4])
    x0.append(popt[2])
    sigmaX0.append(popt[4])
    y0.append(popt[1])
    sigmaY0.append(popt[3])

    print 'Fig %d : (x,y) = (%3.2f,%3.2f), FWHM x = %3.2f, FWHM y  = %3.2f' %(i,x0[i],y0[i],FWHMx[i],FWHMy[i])

    data_fitted.append(fX.TwoDGaussian((x, y), popt[0],popt[1],popt[2],popt[3],popt[4]).reshape(1024, 1280))

    #plot the beamspot
    fig, ax = plt.subplots(1, 1)
    ax.imshow(data[i], cmap=plt.cm.jet,origin='bottom',
        extent=(x.min(), x.max(), y.min(), y.max()))
    ax.contour(x, y, data_fitted[i], 5, colors='w',linewidths=0.8)
    plt.xlim( (popt[2]-4*popt[4], popt[2]+4*popt[4]) )
    plt.ylim( (popt[1]-4*popt[3], popt[1]+4*popt[3]) )
    plt.show()

    #ask if the person wants to acquire a new image to improve the alignement
    pressedkey = raw_input('Do you want to acquire an other image [y (yes) or n (no)]: ')
    if (pressedkey =='n'):
        answer = pressedkey
    #increase i
    i+=1

#stop data acquisition
print('Stopping acquisition...')
cam.stop_acquisition()

#stop communication
cam.close_device()

#convert list to np.array
relativePos = np.array(relativePos)
data = np.array(data)
FWHMx = np.array(FWHMx)
FWHMy = np.array(FWHMy)
x0 = np.array(x0)
y0 = np.array(y0)
sigmaX0 = np.array(sigmaX0)
sigmaY0 = np.array(sigmaY0)

#plot the FWHM vs. relPos
fig, ax = plt.subplots(1,1)
ind = np.argsort(relativePos)
ax.plot(relativePos[ind],(np.sqrt(FWHMx**2+FWHMy**2))[ind])
ax.set_xlabel('Position [mm]')
ax.set_ylabel('FWHM [px]')
ax.grid()
date = datetime.datetime.today()
if not os.path.isdir(plotFolderPath):
    os.makedirs(plotFolderPath)
plt.savefig(plotFolderPath+date.strftime('%Y%m%d%H%M%S')+'FWHM_pos.pdf')
plt.savefig(plotFolderPath+date.strftime('%Y%m%d%H%M%S')+'FWHM_pos.png')


indOfMinFWHM = np.argmin(np.sqrt(FWHMx**2+FWHMy**2))

fig, axarr = plt.subplots(1,np.size(data,0))
#plot all the images besides each other
for iImg in ind:
    axarr[iImg].imshow(data[iImg], cmap=plt.cm.jet,origin='bottom',
        extent=(x.min(), x.max(), y.min(), y.max()))
#    axarr[iImg].contour(x, y, data_fitted[iImg], 5, colors='w',linewidths=0.8)
    axarr[iImg].set_xlim( (x0[iImg]-12, x0[iImg]+12) )
    axarr[iImg].set_ylim( (y0[iImg]-12, y0[iImg]+12) )
    axarr[iImg].set_yticklabels('',visible=False)
    axarr[iImg].set_xticklabels('',visible=False)
    axarr[iImg].set_title('%5.3f mm'%relativePos[iImg],fontsize=8)
    if iImg == indOfMinFWHM:
        axarr[iImg].set_frame_on(True)
        for pos in ['top', 'bottom', 'right', 'left']:
            axarr[iImg].spines[pos].set_edgecolor('r')
            axarr[iImg].spines[pos].set_linewidth(2)
    else:
        axarr[iImg].set_frame_on(False)
plt.show()
date = datetime.datetime.today()

plt.savefig(plotFolderPath+date.strftime('%Y%m%d%H%M%S')+'ImgPSF.pdf')
plt.savefig(plotFolderPath+date.strftime('%Y%m%d%H%M%S')+'ImgPSF.png')


#save data
if not os.path.isdir(dataFolderPath):
    os.makedirs(dataFolderPath)
date = datetime.datetime.today()
np.save(dataFolderPath+date.strftime('%Y%m%d%H%M%S')+'data.npy',data)
np.save(dataFolderPath+date.strftime('%Y%m%d%H%M%S')+'relativePos.npy',relativePos)
\end{lstlisting}

\subsection{AcquisAndSaveXimea.py}
\label{subapp:AcquisAndSaveXimea}

\begin{lstlisting}
#%% Script to acquire images average over nbrImgAveraging images and save them into fits file

from ximea import xiapi
import datetime
import functionsXimea as fX
import winsound
import numpy as np

#%%instanciation --------------------------------------------------------------
#number of image to average
nbrImgAveraging = 5000
numberOfFinalImages = 1

#Cropping information
sizeImg = 256

#Parameter of camera and saving
folderPathCropped = 'data//cropped/20/'
darkFolderPathCropped = '.data/dark//cropped/20/'
folderPathFull = 'data/full/'
darkFolderPathFull = '/data/dark//full/'
nameCamera = 'Ximea'
focusPos = 11.63

#Sound
duration = 1000  # millisecond
freq = 2000  # Hz

#initial guess for the fit depending on the position of the beam in the CCD
initial_guess = [250, 468, 954, 3, 3] # [max PSF,y,x,sigmay,sigmax]

#------------------------------------------------------------------------------
#%% data acquisition ----------------------------------------------------------

#Opening the connection to the camera
cam = xiapi.Camera()
cam.open_device()
cam.set_imgdataformat('XI_MONO8') #XIMEA format 8 bits per pixel
cam.set_gain(0)

img = xiapi.Image()
if cam.get_acquisition_status() == 'XI_OFF':
    cam.start_acquisition()
#%% exposition
cond = 1
while bool(cond):
    source = ''
    winsound.Beep(freq, duration)
    source = int(raw_input('Is the source turned on and at focus point (usually %5.3f mm) (yes = 1) ? '%focusPos))
    if source == 1:
        cond = 0
    else:
        print 'Please turn on the source and place the camera on the focus point (%5.3f mm)'%focusPos

if bool(source):
    #Set exposure time
    cam.set_exposure(fX.determineUnsaturatedExposureTime(cam,img,[1,10000],1))
    #get centroid
    centroid = fX.acquirePSFCentroid(cam,img,initial_guess)
    print 'centroid at (%d, %d)' %(centroid[0],centroid[1])

#%%Acquire images at different camera position

acquire = 1
while bool(acquire):
    cond = 1
    while bool(cond):
        dark = ''
        winsound.Beep(freq, duration)
        dark = int(raw_input('Is the source turned off (yes = 1) ? '))
        if dark == 1:
            cond = 0
        else:
            print 'Please shut down the source.'

    winsound.Beep(freq, duration)
    pos = float(raw_input('What is the position of the camera in mm focused (%5.3f mm) dephase 2Pi (pos+ = %5.3f mm, pos- = %5.3f) ? '%(focusPos,focusPos+3.19,focusPos-3.19)))

    if bool(dark):
        print 'Acquiring dark image...'
        # Acquire dark images
        [darkData,stdDarkData] = fX.acquireImg(cam,img,nbrImgAveraging)
        print 'Cropping'
        [darkdataCropped,stddarkDataCropped] = fX.cropAroundPSF(darkData,stdDarkData,centroid,sizeImg,sizeImg)
        print 'saving'        
        fX.saveImg2Fits(datetime.datetime.today(),darkFolderPathCropped,nameCamera,darkdataCropped,stddarkDataCropped,str(int(np.around(100*(focusPos-pos),0))),nbrImgAveraging)
        fX.saveImg2Fits(datetime.datetime.today(),darkFolderPathFull,nameCamera,darkData,stdDarkData,str(int(np.around(100*(focusPos-pos),0))),nbrImgAveraging)

    #Acquire images -------------------------
    cond = 1
    while bool(cond):
        source = ''
        winsound.Beep(freq, duration)
        source = int(raw_input('Is the source turned on (yes = 1) ? '))
        if source == 1:
            cond = 0
        else:
            print 'Please place turn on the camera'

    if bool(source):
        print 'Acquiring images...'
        # Acquire focused images
        for iImg in range(numberOfFinalImages):
            imgNumber = iImg+1
            print 'Acquiring Image %d'%imgNumber
            [data,stdData] = fX.acquireImg(cam,img,nbrImgAveraging)
            print 'Cropping'
            [dataCropped,stdDataCropped] = fX.cropAndCenterPSF(data-darkData,stdData+stdDarkData,sizeImg,initial_guess)
            print 'Saving'
            fX.saveImg2Fits(datetime.datetime.today(),folderPathCropped,nameCamera,dataCropped,stdDataCropped,str(int(np.around(100*(focusPos-pos),0))),nbrImgAveraging)
            fX.saveImg2Fits(datetime.datetime.today(),folderPathFull,nameCamera,data-darkData,stdData+stdDarkData,str(int(np.around(100*(focusPos-pos),0))),nbrImgAveraging)

    cond = 1
    while bool(cond):
        acquire = ''
        winsound.Beep(freq, duration)
        acquire = int(raw_input('Do you want to acquire at an other camera position (yes = 1, no = 0) ? '))
        if acquire == 1:
            cond = 0
        elif acquire == 0:
            cond = 0
        else:
             print 'please answer with 0 or 1 for no or yes, respectively'


##Stop the acquisition
cam.stop_acquisition()
cam.close_device()

print 'Acquisition finished'

\end{lstlisting}

\subsection{functionsXimea.py}
\label{subapp:functionsXimea}

\begin{lstlisting}
import numpy as np
import pyfits
import os
import scipy.optimize as opt
#%% Functions -----------------------------------------------------------------


#Create and save .fits from numpy array
def saveImg2Fits(date,folderPath,Detector,data,stdData,pos,nbrAveragingImg):

    #date : datetime at which the data where taken
    #folderPath : where to save the data$
    #Detector : name of detector (ex:Ximea)
    #data : np.array containing the image
    #stdData: np.array containing the error on each pixel
    #pos : the position of the camera on the sliding holder in mm

    imgHdu = pyfits.PrimaryHDU(data)
    stdHdu = pyfits.ImageHDU(stdData,name = 'imgStdData')
    hdulist = pyfits.HDUList([imgHdu,stdHdu])

    if not os.path.isdir(folderPath):
        os.makedirs(folderPath)

    hdulist.writeto(folderPath + date.strftime('%Y%m%d%H%M%S')+'_'+Detector+'_'+pos+'.fits')

def acquireImg(cam,img,nbrImgAveraging):
    imgData = np.zeros([1024,1280])
    stdData = np.zeros([1024,1280])
    for iImg in range(nbrImgAveraging) :
        #print iImg
        cam.get_image(img)
        imgTmpData = img.get_image_data_numpy()
        imgData[:,:] += imgTmpData
        stdData += imgTmpData*imgTmpData

    stdData = np.sqrt((stdData-imgData*imgData/nbrImgAveraging)/(nbrImgAveraging-1))/(np.sqrt(nbrImgAveraging))
    imgData = imgData/nbrImgAveraging

    return [imgData,stdData]


def determineUnsaturatedExposureTime(cam,img,exposureLimit,precision):
     exposureTimes = exposureLimit
     if cam.get_acquisition_status() == 'XI_OFF':
         cam.start_acquisition()

     while np.absolute(np.diff(exposureTimes))>precision:
         expTime2check = int(np.round(np.nanmean(exposureTimes)))
         print 'Try expTime : %d [us]\n' %expTime2check
         cam.set_exposure(expTime2check)
         data = acquireImg(cam,img,10)[0]

         if np.sum(data>250)>1:
             exposureTimes[1] = int(np.ceil(np.nanmean(exposureTimes)))
         else:
             exposureTimes[0] = int(np.floor(np.nanmean(exposureTimes)))
         print 'exposure time between %d and %d \n' %(exposureTimes[0],exposureTimes[1])

     return int(np.floor(np.nanmean(exposureTimes)))

def TwoDGaussian((x, y), A, yo, xo, sigma_y, sigma_x):
    g = A*np.exp( - ((x-xo)**2/(2*sigma_x**2) + ((y-yo)**2)/(2*sigma_y**2)))
    return g.ravel()

def acquirePSFCentroid(cam,img,initial_guess):
    #create the matrix grid of the detector CCD

    data = acquireImg(cam,img,200)[0]
    centroid = getPSFCentroid(data,initial_guess)
    return centroid

def getPSFCentroid(data,initial_guess):
    
    x = np.linspace(0,1280,1280)
    y = np.linspace(0,1024,1024)
    x, y = np.meshgrid(x, y)
    print 'fitting'
    popt,pcov = opt.curve_fit(TwoDGaussian, (x,y), data.ravel(), p0 = initial_guess)
    print 'fitting done'
    return [popt[2],popt[1]]


def cropAndCenterPSF(data,stdData,size,initial_guess):
    Xextent = np.size(data,1)
    Yextent = np.size(data,0)

    centroid = getPSFCentroid(data,initial_guess)

    minMarge = np.min([centroid[0],centroid[1],Xextent-centroid[0],Yextent-centroid[1]])

    if minMarge>size/2:
        return cropAroundPSF(data,stdData,centroid,size,size)
    elif minMarge<size/2:
        return cropAroundPSF(data,stdData, centroid,2*minMarge,2*minMarge)


def cropAroundPSF(data,stdData,centroid,sizeX,sizeY):

    pxX = [int(np.floor(centroid[0])-np.ceil(sizeX/2)),int(np.floor(centroid[0])+np.ceil(sizeX/2))]
    pxY = [int(np.floor(centroid[1])-np.ceil(sizeY/2)),int(np.floor(centroid[1])+np.ceil(sizeY/2))]

    dataCropped = data[pxY[0]:pxY[1],pxX[0]:pxX[1]]

    stdDataCropped = stdData[pxY[0]:pxY[1],pxX[0]:pxX[1]]

    return [dataCropped,stdDataCropped]

\end{lstlisting}

