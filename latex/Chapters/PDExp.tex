% Chapter Template

\chapter{Phase Diversity Experiment} % Main chapter title

\label{PDExp} % Change X to a consecutive number; for referencing this chapter elsewhere, use \ref{ChapterX}



%----------------------------------------------------------------------------------------
%	SECTION 1
%----------------------------------------------------------------------------------------

\section{Theoretical Background}



%----------------------------------------------------------------------------------------
%	SECTION 2
%----------------------------------------------------------------------------------------

\section{Experimental Setup}

The design of the experiment was already done by \citet{Bouxin_PDM}. I built the setup according to her plans and specification.

The experiment is mounted on a pressurized legs optical table. The setup contains six components : a light source, an entrance pupil, an imaging system, a converging lens to focus the beam on the camera, a camera and a wavefront sensor.

%------------------------------------------------------------------------------
%	SECTION 3
%----------------------------------------------------------------------------

\section{Results}

This section present the results of the phase diversity experiment, with the introduction of different sources of aberration.

\subsection{Astigmatism}

The first aberration studied in this work is the astigmatism aberration introduced by a tilted parallel plane plate (link to section). A parallel plane plate introduces astigmatism in addition to the defocus introduced by the perpendicular plate.
