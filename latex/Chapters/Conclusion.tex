
\chapter*{Conclusion}
\addcontentsline{toc}{chapter}{Conclusion}
\label{Conclusion}

The wavefront reconstruction is at the forefront of the astronomical instruments development nowadays. Having bigger and bigger telescopes will not bring a lot without methods to monitor and correct the perturbations introduced by the atmosphere. Both the VLT and the ELT as well as the DAG telescope use and will use wavefront sensing and reconstruction to correct most of the aberrations and improve the resolution up to the diffraction limit. At the moment, the phase diversity method is still not widely used but its advantages are recognized regarding the NCPA and its implementation simplicity.

\vspace{1cm}

We conducted an experiment to test the phase diversity algorithm developed at ONERA by \citet{mugnier_2006}. The algorithm's tests regarding noise level and $j_{max}$ are interesting. We discovered that, as expected, the noise level in the PSFs is important especially in order to reconstruct wavefronts, and that $j_{max}$ modifies significantly the wavefront retrieved.

We then computed a Zemax model of our optical system to introduce calibrated astigmatism with a parallel plane plate in Plexiglas. The results are not as expected. The astigmatism retrieved by the phase diversity follows the right behaviour qualitatively, but there is an offset. We also compared the vertical coma retrieved with the Zemax model, and once again the results were off. This shows that we might have an alignment problem.

Two problem remains, first the offset between the Zemax model of the parallel plane plate and the phase diversity. And then, the difference between the Shack-Hartmann and the phase diversity. The first problem shows that either the Zemax model does not simulate correctly the optical bench; or the ONERA phase diversity algorithm does not respect the linearity of the aberrations introduction. The second problem could be due to a misalignment of the Shack-Hartmann, because the alignment procedure we used is perturbed by the aberrations present in the system.

In the future, we would need to take a step back and be sure that we have a way to characterize the system aberrations before trying to use the phase diversity. The alignment of the source and the collimating f=200mm lens should be simpler using mounting system compatible with the 4-rods. And the Shack-Hartmann position could be check using a rule placed at the entrance pupil.

\vspace{1cm}

After having tested the ONERA algorithm, the aim was to develop our own phase diversity. It works as expected. And the behaviour with respect to noise and $j_{max}$ are satisfying. An interesting result is the overestimation and underestimation of the Zernike coefficients when the number of Zernike used by the retrieval is smaller and bigger, respectively, than the one used to create the simulated PSFs. This overestimation and underestimation might come from the fact that the algorithm counter-balances for the loss or gain of available Zernike components as there is a definite budget of aberrations to spread over a definite number of Zernike. The validity domain of the method is limited as the method is based on a strong approximation. The method is usable up to a wavefront RMS error of 50 nm. Above the difference becomes too important to be trustworthy. 

To increase the validity domain of the algorithm, a recursive approach has been tested. This method could be used on a system having a deformable mirror. Th results are really satisfying and we succeeded to multiply by three the validity domain the analytical phase diversity. This is similar to the validity domain of the ONERA algorithm, with the advantage that our algorithm is based on simpler principles and we understand fully the way it works.

Now, before implementing this algorithm on any telescope, we would need to first test it on real PSFs and characterize it as we did for the ONERA algorithm. This could be done after having improved the phase diversity experiment as said above.

\vspace{1cm}

In conclusion, this work was really interesting and I learned the hard way that in science research most of the time it does not go as planned. The experimental work was a challenge, as it was the first time I worked as much in a lab. The development of the phase diversity algorithm was a bit more common. But I learned a lot about discrete Fourier transform and their specificity. It has been a great experience. There is still a lot to do especially improving the laboratory experiment in order to correctly test the ONERA algorithm and more importantly test our own phase diversity on real data. 